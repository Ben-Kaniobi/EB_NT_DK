%%%%%%%%%%%%%%%%%%%%%%%%%%%%%%%%%%%%%%%%%%%%%%%%%%%%%%%%%%%%%%%%%%%%%%%%%%%%%%%
% Titel:   Anforderungen
% Autor:   Nicola K�ser
%%%%%%%%%%%%%%%%%%%%%%%%%%%%%%%%%%%%%%%%%%%%%%%%%%%%%%%%%%%%%%%%%%%%%%%%%%%%%%%
\chapter{\PA{2}}\label{ch:pa2}
Ein Teil der Zeit in der \PA{2} wurde weiter f�r die Naherkennung aufgewendet, einerseits um die im \autoref{s:pendenzen} beschriebenen Punkte zu realisieren und anderseits weil es durch verschiedene �nderungen bei anderen Bereichen auch bei der Naherkennung zu �nderungen kam. In diesem Kapitel werden diese Anpassungen und Erweiterungen erl�utert.
%
	\section{Anpassungen}\label{s:anpassungen}
	%
		\subsection{Platzierung der Senseoren}\label{ss:platzierung}
		Da sich in der \PA{2} beim Volumenkonzept �nderungen ergaben, konnte die Naherkennung nicht wie geplant montiert werden. Statt drei von den Infrarotsensoren vorne und einer hinten zu verwenden, wurde die Anordnung so ge�ndert, dass nun zwei vorne und zwei hinten platziert sind. \missingfigure{Neue Position, siehe AJ 16.4.14} Dies wurde auch so in der Software ge�ndert. Bei der Platzierung der Ultraschallsensoren hat sich nichts ge�ndert.
		%
		\par In der neue Anordnung werden nur noch Infrarotsensor-Printe der Quadratischen Variante eingesetzt. Dies trifft auch f�r den grossen Roboter zu. Aus diesem Grund und zur Redundanz wurden weitere \todo{4?}vier Printe dieser Variante gefertigt, sodass jetzt noch \todo{2?}zwei davon als Ersatz vorhanden sind.
		%
		\subsection{Software�nderungen bez�glich IR-Sensoren}\label{ss:softchangeir}
		Um die Software effizienter zu gestalten wurde auf den Task f�r Infrarotsensoren verzichtet. Das Einlesen der Sensoren erfolgt nun nicht mehr durch Polling der GPIO-Pins, sondern es wird bei einer Flanken�nderung am Pin ein Interrupt ausgel�st. Im Interrupt-Handler wird dann bestimmt ob es sich um eine positive oder negative Flanke handelt und dementsprechend die Alarmflag-Variable gesetzt. Es w�re auch m�glich gewesen lediglich die Variable zu toggeln, da ja nach einer positiven Flanke immer eine negative Flanke folgen sollte. Auf diese Methode wurde aber verzichtet, damit die Software m�glichst fehlerresistent ist.
		%
		\par Um die Modularit�t der Software beizubehalten, wurde eine neue C-Datei erstellt, in der sich alle Interrupt-Handler f�r externe Interrupts befinden. Beispielsweise sind zum Teil einige \textit{exti-lines\footnote{"`Kan�le"' f�r externe Interrupts}} in einem Handler zusammengefasst. So kann es sein, dass mehrere Module den selben Handler ben�tigen. Damit es desshalb kein Durcheinander gibt wird nun in den Handlern pro \textit{exti-line} einfach eine Funktion des jeweiligen Moduls aufgerufen.
		\image{content/image/uebersicht_interrupthandler}{scale=.6}[Das Interrupt-Handler-Modul ruft Funktionen der Naherkennung auf][Interrupt-Handler][pic:interrupthandler]
		%
		\subsection{Software�nderungen bez�glich Ultraschallsensoren}\label{ss:softchangeus}
		Wie bei den Tests in der \PA{1} festgestellt und in \autoref{s:pendenzen} erl�utert wurde, gab es ein Problem mit der \iic-Schnittstelle. Beim Fehlen eines angesprochenen Slaves blieb der Bus-Zustand permanent auf \textit{Busy}. In der neuen Version wurde eine zus�tzliche Funktion implementiert, die ein Teil der \iic-Initialisierung wiederholt. Dadurch wird der Buszustand zur�ckgesetzt und die Schnittstelle kann wieder benutzt werden. W�hrend der \PA{1} wurde das \iic-Modul vom RoboBoard so erweitert, dass der Fehlzustand mit Hilfe einer Timeout-Variable erkannt werden konnte worauf die Variable \code{i2c\_timeout\_flag} gesetzt wurde. Mit der aktuellen Version kann nun also das \iic-Problem behoben werden indem nach jeder �bertragung diese Variable �berpr�ft und im Fehlerfall die Funktion \code{reinitI2C()} ausgef�rd wird.
		%
	\section{Erweiterungen}\label{s:erweiterungen}
	%
		\subsection{Softwarerg�nzungen bez�glich Ultraschallsensoren}\label{ss:softaddus}
		Da die Ultraschallmessung fehleranf�lliger als die Infrarotmessung ist, wurde eine �berpr�fung der Messwerte in der Software hinzugef�gt. %Nach einer Messung wird nun der aktuelle Status zuerst mit den letzten zwei Status verglichen. Wenn zwei oder alle der drei Werten 
		
		%der Messwert immer noch direkt zu einem bin�ren Wert, mit der Bedeutung "`Objekt zu nah"' oder "`kein Objekt zu nah"', umgewandelt. Danach wird dieser Wert mit den letzten zwei 
		%
		
		
		
		