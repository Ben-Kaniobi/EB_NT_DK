%%%%%%%%%%%%%%%%%%%%%%%%%%%%%%%%%%%%%%%%%%%%%%%%%%%%%%%%%%%%%%%%%%%%%%%%%%%%%%%
% Titel:   Evaluation
% Autor:   Nicola K�ser
%%%%%%%%%%%%%%%%%%%%%%%%%%%%%%%%%%%%%%%%%%%%%%%%%%%%%%%%%%%%%%%%%%%%%%%%%%%%%%%
\chapter{Evaluation}\label{ch:evaluation}
Mit Hilfe der Liste konnte entschieden werden, welche Sensoren f�r die Evaluation bestellt werden. Einerseits wurden diverse Infrarot-Sensoren ausgew�hlt, anderseits auch Ultraschall-Module. Ein Laser-Sensor der in einer Eurobot-Thesis verwendet wurde, wurde auch noch dazu genommen. So konnte ein Einblick in alle erl�uterten  geeigneten Methoden gewonnen werden und die Vor- und Nachteile abgewogen werden.
%
	\section{GP2D120 und GP2Y0A02YK}\label{s:gp2d120-und-gp2y0a02yk}
	Die Infrarot-Sensoren \ttodo{GP2Y0A21 und GP2Y0A21} von Sharp basieren auf dem Triangulationsprinzip. \todo{nur einer ist nachfolger}Es sind Nachfolgemodelle der Typen wie die fr�heren Teams sie verwendeten.
	\par\todo{Diagramm}
	Die im vorherigen Kapitel beschriebenen Probleme wie Aussetzer und Messfehler konnten nicht reproduziert werden. Es gab einige Abweichungen wegen spiegelnder Oberfl�che der Objekte, dieses Ph�nomen ist aber gut nachvollziehbar. Im Diagramm ist jedoch zu erkennen, dass auch bei diesen Sensor-Versionen die in der Thesis beschriebene Problematik der zweideutigen Messresultate \cite{lit:gegnerischer_roboter}  besteht.
	\section{IS471F}\label{s:is471f}
	Auch von Sharp, ist der Sensor IS471F. Dabei handelt es sich um ein Bauteil, das in einer einfachen Schaltung als Distanzschalter funktioniert.
	\image{content/image/is471f}{scale=.5}[Einfache Beispielschaltung mit Sensor IS471F \cite{pic:is471f}][Einfache Beispielschaltung mit Sensor IS471F][pic:is471f]
	Die im Sensor integrierte Schaltung steuert externe Infrarot-LEDs an. Ist das so modulierte Licht f�r den Sensor sichtbar, wo liefert dieser einen definierten Pegel, andernfalls den invertierten Pegel. Dies funktioniert nicht nur bei direkter Bestrahlung, wie bei Lichtschranken, sondern auch bei Reflektiertem Licht. F�r die Naherkennung k�nnte also �ber die Lichtst�rke die Schaltdistanz eingestellt werden.
	%
	\section{SRF-Module von Devantech}\label{s:srf-module-von-devantech}
	Devantech\footnote{\url{http://www.devantech.co.uk/}} entwickelt diverse elektronische Module, darunter befindet sich auch eine Reihe von Ultraschall-Modulen zur Distanzmessung.
	Die \ttodo{blablabla} bla von 
		\subsection{SRF02}
		\subsection{SRF08}
		\subsection{SRF10}
	\section{Lasersensor von Baumer}\label{s:lasersensor-von-baumer}
	Der Sensor OADM 13U7480/S35A von Baumer\footnote{\url{http://www.baumer.com/ch-de/}}