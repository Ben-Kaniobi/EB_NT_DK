%%%%%%%%%%%%%%%%%%%%%%%%%%%%%%%%%%%%%%%%%%%%%%%%%%%%%%%%%%%%%%%%%%%%%%%%%%%%%%%
% Titel:   Evaluation
% Autor:   Nicola K�ser
%%%%%%%%%%%%%%%%%%%%%%%%%%%%%%%%%%%%%%%%%%%%%%%%%%%%%%%%%%%%%%%%%%%%%%%%%%%%%%%
\chapter{Evaluation}\label{ch:evaluation}
Mit Hilfe der Liste konnte entschieden werden, welche Sensoren f�r die Evaluation bestellt werden. Einerseits wurden diverse Infrarot-Sensoren ausgew�hlt, anderseits auch Ultraschall-Module. Ein Laser-Sensor der in einer Eurobot-Thesis verwendet wurde, wurde auch noch dazu genommen. So konnte ein Einblick in alle erl�uterten  geeigneten Methoden gewonnen werden und die Vor- und Nachteile abgewogen werden.
%
	\section{GP2D120 und GP2Y0A02YK}\label{s:gp2d120-und-gp2y0a02yk}
	Die Infrarot-Sensoren GP2D120 und GP2Y0A02YK von Sharp basieren auf dem Triangulationsprinzip. Es sind Nachfolgemodelle der Typen wie die fr�heren Teams sie verwendeten.
	\begin{figure}[H]%H htbp
		\centering
		%%%%%%%%%%%%%%%%%%%%%%%%%%%%%%%%%%%%%%%%%%%%%%%%%%%%%%%%%%%%%%%%%%%%%%%%%%%%%%%
% Titel:   Diagramm: GP2D120
% Autor:   Nicola K�ser
%%%%%%%%%%%%%%%%%%%%%%%%%%%%%%%%%%%%%%%%%%%%%%%%%%%%%%%%%%%%%%%%%%%%%%%%%%%%%%%
\begin{tikzpicture}
	\begin{axis}[
		height=9cm, width=13cm,
		xlabel=Distanz/m, ylabel=U/V,
		grid=major,
		legend style={cells={anchor=west}}]
	%
	%\addplot[color=blue,mark=*] coordinates {
	\addplot coordinates {
		( 10, 2.32)
		( 12, 2.7 )
		( 15, 2.73)
		( 18, 2.6 )
		( 20, 2.52)
		( 30, 1.94)
		( 40, 1.48)
		( 50, 1.18)
		( 60, 0.98)
		( 80, 0.73)
		(100, 0.59)
		(140, 0.37)
	};
	\addlegendentry{Holz}
	%
	%\addplot[color=red,mark=square*] coordinates {
	\addplot coordinates {
		( 10, 2.73)
		( 12, 2.8 )
		( 15, 2.57)
		( 18, 2.3 )
		( 20, 2.2 )
		( 30, 1.73)
		( 40, 1.44)
		( 50, 1.14)
		( 60, 1.03)
		( 80, 0.79)
		(100, 0.66)
		(140, 0.5 )
	};
	\addlegendentry{Metall, gl�nzend}
	%
	%\addplot[color=teal,mark=triangle*] coordinates {
	\addplot coordinates {
		( 10, 2.52)
		( 12, 2.8 )
		( 15, 2.78)
		( 18, 2.62)
		( 20, 2.52)
		( 30, 1.96)
		( 40, 1.5 )
		( 50, 1.2 )
		( 60, 1   )
		( 80, 0.75)
		(100, 0.58)
		(140, 0.4 )
	};
	\addlegendentry{Metall, matt}
	%
	\end{axis}
\end{tikzpicture}
		\caption[Messungen mit dem IR-Sensor GP2D120]{Messungen mit dem Infrarotsensor GP2D120}
		\label{img:messung-gp2d120}
	\end{figure}
	\begin{figure}[H]%H htbp
		\centering
		\begin{tikzpicture}
	\begin{axis}[
		height=9cm, width=13cm,
		xlabel=Distanz/m, ylabel=U/V,
		grid=major,
		legend style={cells={anchor=west}}]
	%
	%\addplot[color=blue,mark=*] coordinates {
	\addplot coordinates {
		( 1  , 2.05)
		( 1.5, 1.86)
		( 2  , 2.65)
		( 2.5, 3.06)
		( 2.7, 3.06)
		( 3  , 2.96)
		( 3.5, 2.82)
		( 4  , 2.54)
		( 5  , 2.13)
		( 6  , 1.85)
		( 7  , 1.62)
		( 8  , 1.44)
		(10  , 1.14)
		(15  , 0.77)
		(20  , 0.57)
		(30  , 0.34)
		(40  , 0.25)
	};
	\addlegendentry{Holz}
	%
	%\addplot[color=red,mark=square*] coordinates {
	\addplot coordinates {
		( 1  , 2.76 )
		( 1.5, 2.64 )
		( 2  , 2.56 )
		( 2.5, 2.57 )
		( 2.7, 2.54 )
		( 3  , 2.44 )
		( 3.5, 2.327)
		( 4  , 2.11 )
		( 5  , 1.84 )
		( 6  , 1.56 )
		( 7  , 1.39 )
		( 8  , 1.28 )
		(10  , 1.05 )
		(15  , 0.74 )
		(20  , 0.57 )
		(30  , 0.44 )
		(40  , 0.34 )
	};
	\addlegendentry{Metall, gl�nzend}
	%
	%\addplot[color=teal,mark=triangle*] coordinates {
	\addplot coordinates {
		( 1  , 2.69)
		( 1.5, 2.22)
		( 2  , 2.67)
		( 2.5, 3.02)
		( 2.7, 3.06)
		( 3  , 3.01)
		( 3.5, 2.83)
		( 4  , 2.51)
		( 5  , 2.13)
		( 6  , 1.85)
		( 7  , 1.63)
		( 8  , 1.47)
		(10  , 1.2 )
		(15  , 0.81)
		(20  , 0.61)
		(30  , 0.37)
		(40  , 0.27)
	};
	\addlegendentry{Metall, matt}
	%
	\end{axis}
\end{tikzpicture}
		\caption[Messungen mit dem IR-Sensor GP2Y0A02YK]{Messungen mit dem Infrarotsensor GP2Y0A02YK}
		\label{img:messung-gp2y0a02yk}
	\end{figure}
	Die im vorherigen Kapitel beschriebenen Probleme wie Aussetzer und Messfehler konnten nicht reproduziert werden. Es gab einige Abweichungen wegen spiegelnder Oberfl�che der Objekte, dieses Ph�nomen ist aber gut nachvollziehbar. Im Diagramm ist jedoch zu erkennen, dass auch bei diesen Sensor-Versionen die in der Thesis beschriebene Problematik der zweideutigen Messresultate \cite{lit:gegnerischer_roboter}  besteht.
	\section{IS471F}\label{s:is471f}
	Auch von Sharp, ist der Sensor IS471F. Dabei handelt es sich um ein Bauteil, das in einer einfachen Schaltung als Distanzschalter funktioniert.
	\image{content/image/is471f}{scale=.5}[Einfache Beispielschaltung mit Sensor IS471F \cite{pic:is471f}][Einfache Beispielschaltung mit Sensor IS471F][pic:is471f]
	Die im Sensor integrierte Schaltung steuert externe Infrarot-LEDs an. Ist das so modulierte Licht f�r den Sensor sichtbar, wo liefert dieser einen definierten Pegel, andernfalls den invertierten Pegel. Dies funktioniert nicht nur bei direkter Bestrahlung, wie bei Lichtschranken, sondern auch bei Reflektiertem Licht. F�r die Naherkennung k�nnte also �ber die Lichtst�rke die Schaltdistanz eingestellt werden.
	\begin{figure}[H]%H htbp
		\centering
		%%%%%%%%%%%%%%%%%%%%%%%%%%%%%%%%%%%%%%%%%%%%%%%%%%%%%%%%%%%%%%%%%%%%%%%%%%%%%%%
% Titel:   Diagramm: IS471F
% Autor:   Nicola K�ser
%%%%%%%%%%%%%%%%%%%%%%%%%%%%%%%%%%%%%%%%%%%%%%%%%%%%%%%%%%%%%%%%%%%%%%%%%%%%%%%
\begin{tikzpicture}
	\begin{axis}[
		ybar,
		bar width=5mm,
		height=4.095cm, width=9cm,
		ylabel=Distanz/m,
		ymajorgrids=true,
		enlargelimits=0.5,
		legend style={
			cells={anchor=west},
			legend pos=outer north east
		},
		xtick=data,
		nodes near coords,
		nodes near coords align={vertical},
		symbolic x coords={Normale LED, Helle LED}]
	%
	\addplot coordinates{(Normale LED, 17) (Helle LED, 35)};
	\addplot coordinates{(Normale LED, 30) (Helle LED, 47)};
	\addplot coordinates{(Normale LED, 40) (Helle LED, 52)};
	\addplot coordinates{(Normale LED, 27) (Helle LED, 37)};
	\addplot coordinates{(Normale LED,  6) (Helle LED,  9)};
	\legend{Holz, {Metall, matt}, {Metall, gl�nzend}, {Metall, eloxiert schwarz}, {Plastik, matt schwarz}}
	%
	\end{axis}
\end{tikzpicture}
		\caption[Messungen mit dem IR-Sensor IS471F]{Messungen mit dem Infrarotsensor IS471F}
		\label{img:messung-is471f}
	\end{figure}
	\todo{auswertung diagramm}
	%
	\section{SRF-Module von Devantech}\label{s:srf-module-von-devantech}
	Devantech\footnote{\url{http://www.devantech.co.uk/}} entwickelt diverse elektronische Module, darunter befindet sich auch eine Reihe von Ultraschall-Modulen zur Distanzmessung.
	Die \ttodo{blablabla} bla von 
		\subsection{SRF02}
		\subsection{SRF08}
		\subsection{SRF10}
	\section{Lasersensor von Baumer}\label{s:lasersensor-von-baumer}
	Der Sensor OADM 13U7480/S35A von Baumer\footnote{\url{http://www.baumer.com/ch-de/}}