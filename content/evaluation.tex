%%%%%%%%%%%%%%%%%%%%%%%%%%%%%%%%%%%%%%%%%%%%%%%%%%%%%%%%%%%%%%%%%%%%%%%%%%%%%%%
% Titel:   Evaluation
% Autor:   Nicola K�ser
%%%%%%%%%%%%%%%%%%%%%%%%%%%%%%%%%%%%%%%%%%%%%%%%%%%%%%%%%%%%%%%%%%%%%%%%%%%%%%%
\chapter{Evaluation}\label{ch:evaluation}
Mit Hilfe der Liste konnte entschieden werden, welche Sensoren f�r die Evaulation bestellt werden. Einerseits wurden Ultraschall-Module ausgew�hlt, anderseits auch diverse Infrarot-Sensoren. Ein Laser-Sensor der in einer Eurobot-Thesis verwendet wurde, wurde auch noch dazu genommen. So konnte ein Einblick in alle erl�uterten  geeigneten Methoden gewonnen werden und die Vor- und Nachteile abgewogen werden.
%
	\section{GP2Y0A21 und GP2Y0A21}\todo{name}
	Die Infrarot-Sensoren GP2Y0A21 und GP2Y0A21 von Sharp basieren auf dem Triangulationsprinzip. Es sind Nachfolgemodelle der Typen wie die fr�heren Teams sie verwendeten.
	\par\todo{Diagramm}
	Die im vorherigen Kapitel beschriebenen Probleme wie Aussetzer und Messfehler konnten nicht reproduziert werden. Es gab einige Abweichungen wegen spiegelnder Oberfl�che der Objekte, dieses Ph�nomen ist aber gut nachvollziehbar. Im Diagramm ist jedoch zu erkennen, dass auch bei diesen Sensor-Versionen die in der Thesis beschriebene Problematik der zweideutigen Messresultate \cite{lit:gegnerischer_roboter}  besteht.
	\section{IS471F}\todo{name}
	Auch von Sharp ist der Sensor IS471F\todo{name}. Dabeil handelt es sich um ein Bauteil, das in einer einfachen Schaltung als Distanzschalter funktioniert.
	\par\todo{Schaltung aus Internet}
	Die im Sensor integrierte Schaltung steuert ein beliebiges LED \todo{bla}
	\section{SRF02}
	\section{SRF08}
	\section{SRF10}
	\section{Baumer nnn}\todo{name}
