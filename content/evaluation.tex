%%%%%%%%%%%%%%%%%%%%%%%%%%%%%%%%%%%%%%%%%%%%%%%%%%%%%%%%%%%%%%%%%%%%%%%%%%%%%%%
% Titel:   Evaluation
% Autor:   Nicola K�ser
%%%%%%%%%%%%%%%%%%%%%%%%%%%%%%%%%%%%%%%%%%%%%%%%%%%%%%%%%%%%%%%%%%%%%%%%%%%%%%%
\chapter{Evaluation}\label{ch:evaluation}
Bla
%
	\section{GP2Y0A21}\todo{name, evtl mit nachfolgendem Thema zusammen nehmen}
	\section{GP2Y0A21}\todo{name}
	%war auch der Kennlinienverlauf des Sensors. Dieser weist zu Beginn des Messbereiches einen Knick auf, wodurch in
	%einem gewissen Distanzbereich eine Zweideutigkeit auftritt und somit eine nicht eindeutig denierte Distanz zuruckliefert.
	\section{IS471F}\todo{name}
	\section{SRF02}
	\section{SRF08}
	\section{SRF10}
	\section{Baumer nnn}\todo{name}
