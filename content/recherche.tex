%%%%%%%%%%%%%%%%%%%%%%%%%%%%%%%%%%%%%%%%%%%%%%%%%%%%%%%%%%%%%%%%%%%%%%%%%%%%%%%
% Titel:   Recherche
% Autor:   Nicola K�ser
%%%%%%%%%%%%%%%%%%%%%%%%%%%%%%%%%%%%%%%%%%%%%%%%%%%%%%%%%%%%%%%%%%%%%%%%%%%%%%%
\chapter{Recherche}\label{ch:recherche}
Zu Beginn mussten Informationen beschafft werden mit welchen Methoden eine Distanzmessung realisiert werden kann. Diese konnten dann in zwei Kategorien eingeordnet werden, eine Kategorie die sich f�r den Eurobot-Wettbewerb eignet und eine Kategorie die aus verschiedenen Gr�nden weniger gut geeignet ist.
%
	\section{Ungeeignete Sensormethoden}
	In diese Kategorie geh�ren Kontaktsensoren, da die Naherkennung dazu dienen soll den Kontakt mit dem Gegner zu verhindern. Weiter dazu geh�ren Kamerasysteme, da f�r die Verarbeitung der Daten sehr viel Systemleistung aufgewendet werden muss, und eine gute Software mehr Zeit und Aufwand erfordert als f�r diese Arbeit geplant ist. Auch Induktive- und kapazitive Sensoren eignen sich f�r die Naherkennung nicht, da diese nur Distanzen im Millimeterbereich messen.
	\par Ein Radarsysteme, wie sie beispielsweise f�r automatische T�ren verwendet werden, w�re theoretisch gut geeignet. Besonders die Tatsache, dass damit auch die Geschwindigkeit des Gegnerroboters ermittelt werden k�nnte w�re von Vorteil. Es ist jedoch so, dass die erh�ltlichen Radarsensoren jeweils eine Reichweite erst ab ca. zwei Metern aufweisen.
	%
	\section{Geeignete Sensormethoden}
	Die geeigneten Sensortypen kann man wiederum in drei Unterkategorien mit unterschiedlichen Vor- und Nachteilen unterordnen. \todo{Bilder}
	%
		\subsection{Infrarot}
		Die meisten Infrarotbasierten Sensoren sind auf dem Prinzip der Triangulation aufgebaut. Eine Lichtquelle im Sensor, meistens eine LED, strahlt Infrarotlicht aus. Trifft das Licht auf ein Objekt, so wird es  in einem zur Distanz im Verh�ltnis stehenden Winkel zur�ckgeworfen. Beim Sensor trifft dieses Licht auf ein Array von Infrarot-Empf�ngern, wodurch der Winkel und somit auch die Distanz zum Objekt berechnet werden kann.
		\image{content/image/ir_prinzip}{scale=.3}[Prinzip der Infrarot-Distanzmessung \cite{pic:ir}][Prinzip der Infrarot-Distanzmessung][pic:ir]
		\par Die fr�heren Eurobot-Teilnehmer der Berner Fachhochschule verwendeten haupts�chlich Infrarotsensoren von Sharp\footnote{http://www.sharp.ch/\todo{ref}} die dieses Prinzip verwenden. Sie stellten damit jedoch einige Probleme fest, so kam es w�hrend dem Wettbewerb vermehrt zu Aussetzern und Fehlmessungen. \cite{lit:gegnerischer_roboter} Diese Fehlmessungen wurden vermutlich wegen St�rungen durch den Autofokus von Fotoapparaten hervorgerufen. Mit moduliertem Infrarotlicht k�nnten solche St�rungen beseitigt werden. M�glicherweise ist dies jedoch mit einem Array von Infrarot-Empf�ngern schwieriger zu realisieren, denn die Sensoren die mit einer Modulation arbeiten greifen nicht auf die Triangulation zur�ck.
		\par So ist eine weitere M�glichkeit die bei Infrarot angewendet wird die Distanzermittlung mittels der Intensit�t. Die Sensoren die auf diesem Prinzip basieren, messen aber f�r gew�hnlich nicht direkt die Distanz, sondern haben einfach eine bestimmte Schaltschwelle.
		\par Mit Infrarot w�re auch noch eine Messung per Laufzeitverfahren m�glich, dazu wurden aber keine Sensoren gefunden.
		%
		\subsection{Ultraschall}
		Da Schall viel langsamer ist als Licht, wird die Laufzeitmessung mit Ultraschall h�ufiger eingesetzt. Ein Ultraschallgeber sendet ein Signal aus, welches dann von einem Objekt reflektiert wird. Mit einem Ultraschallsensor kann das Signal dann wieder empfangen werden. Durch die halbe Zeitdifferenz und die Schallgeschwindigkeit kann dann die Distanz berechnet werden. Je nach Frequenz des Ultraschalls variiert die Gr�sse des Messkegels. Der Vorteil von Ultraschallsensoren im Gegensatz zu Infrarotsensoren ist, dass das Material und die Farbe des zu detektierenden Objekts keinen Einfluss auf die Messung hat. In der Thesisarbeit hat ein Vorg�ngerteam beim Evaluieren verschiedener Sensoren Probleme mit bewegten Objekten aufgrund des Dopplereffektes\footnote{Zeitliche Stauchung/Dehnung eines Signals bei schneller Ver�nderungen des Abstands zwischen Sender und Empf�nger, was zur �nderung der wahrgenommenen Frequenz f�hrt.} festgestellt. \cite{lit:gegnerischer_roboter}
		%
		\subsection{Laser}
		F�r m�glichst genaue Messungen verwendet man am Besten Laserensoren. Die meisten Lasersensoren nutzen auch die Triangulation, im Gegensatz zu Sensoren mit Infrarot-LEDs erh�lt man aber aufgrund der Lasereigenschaften eine viel genauere Messung. Der Nachteil liegt auf der Hand, durch die genauere Messung ist der Messpunkt auch sehr viel kleiner. M�chte man einen gr�sseren Bereich abdecken, m�ssen desshalb entweder mehrere Laser verwendet werden, oder es muss mittels mechanischer Konstruktion an verschiedenen stellen Mehrmals gemessen werden. \todo{evtl. Bild b) von SLAM-Buch auf Seite 46}
		%
	\section{Sensorliste}
	Mit der oben genannten Kategorisierung wurde eine Liste von geeigneten erh�ltlichen Sensoren zusammengestellt:
	%%%%%%%%%%%%%%%%%%%%%%%%%%%%%%%%%%%%%%%%%%%%%%%%%%%%%%%%%%%%%%%%%%%%%%%%%%%%%%%
% Titel:   Sensortabelle
% Autor:   Nicola K�ser
%%%%%%%%%%%%%%%%%%%%%%%%%%%%%%%%%%%%%%%%%%%%%%%%%%%%%%%%%%%%%%%%%%%%%%%%%%%%%%%
% PDF-Seite 90� rotieren
\begin{rotpdf90}
\begin{table}[htbp]
	% Lokale Commands (nur g�ltig bis \end{table})
	\newcommand{\T}[1]{\textcolor{white}{\textbf{#1}}}  % Titel
	\newcommand{\n}[2][c]{\begin{tabular}[#1]{@{}c@{}}#2\end{tabular}}  % Spezielle Zelle in der Umruch (\\) verwendet werden kann
	\newcommand{\mc}[2]{\multicolumn{#1}{c|}{#2}}  % Mehrere Zeilen in einer Zelle
	\newcommand{\<}{\textless}
	%
	\centering
	\rotatebox{90}{
	\begin{tabular}{|l|S[table-format=1.1]cS[table-format=1.2]|c|c|lS[table-format=1.1]|l|S[table-format=1.1]|}
		\hline \rowcolor{bfhblue}
		\T{Bezeichnung} & \mc{3}{\T{Bereich/m}} & \T{Output} & \T{Vcc/V} & \mc{2}{\T{I/mA}} & \T{Methode} & \T{Preis (ca.)} \\ \hline
		FHCK 07P6901/KS35A   & 0.1  & - & 0.6  & Schaltschwelle    & 10 - 30  &  &    20 & Infrarot          &  160 �   \\ \hline
		FHDK 10P1101/KS35    & 0.2  & - & 1.2  & Schaltschwelle    & 10 - 30  &  &    20 & Infrarot          &  130 \$  \\ \hline
		GP2D02               & 0.1  & - & 0.8  & Seriell 8-Bit     & 4.4 - 7  &  &    22 & Infrarot          &   13 \$  \\ \hline
		GP2D120              & 0.04 & - & 0.3  & 2.6 V - 0.4 V     &  5       &  &    33 & Infrarot          &   11 �   \\ \hline
		GP2Y0A02YK           & 0.2  & - & 1.5  & 2.6 V - 0.4 V     &  5       &  &    33 & Infrarot          &   17 �   \\ \hline
		GP2Y0A21             & 0.1  & - & 0.8  & 2.6 V - 0.4 V     &  5       &  &    30 & Infrarot          &   14 �   \\ \hline
		GP2Y0A41             & 0.04 & - & 0.3  & 2.7 V - 0.4 V     &  5       &  &    12 & Infrarot          &   20 Fr. \\ \hline
		IS471F           & \mc{3}{je nach LED} & Schaltschwelle    & 4.5 - 16 &  &   3.5 & Infrarot          &    2.5 � \\ \hline
		Hokuyo URG-04LX      & 0.06 & - & 4.1  & USB 2.0, RS-232   &  5       &  &   500 & Laser (Cl. 1)     & 1600 �   \\ \hline
		O1D100               & 0.2  & - & 10   & 2 Schaltschwellen & 18 - 30  & \< & 150 & Laser (Cl. 2)     &  640 Fr. \\ \hline
		O1D102               & 0.2  & - & 3.5  & 2 Schaltschwellen & 18 - 30  & \< & 150 & Laser (Cl. 2)     &  640 Fr. \\ \hline
		OADM 13U7480/S35A    & 0.05 & - & 0.55 & 0 V - 10 V        & 12 - 28  & \< &  80 & Laser (Cl. 2)     &  780 �   \\ \hline
		VDM28-8-L-IO/37c/122 & 0.2  & - & 8    & IO-Link           & 18 - 30  &  &       & Laser (Cl. 2)     &  530 Fr. \\ \hline
		VDM28-8-L-IO/37c/136 & 0.2  & - & 8    & IO-Link           & 18 - 30  &  &       & Laser (Cl. 2)     &  510 Fr. \\ \hline
		SEN13635B            & 0.03 & - & 4    & \iic              &  5       &  &    15 & US (40 kHz, 30�)  &   12 \$  \\ \hline
		SRF02                & 0.01 & - & 4    & \iic , RS-232     &  5       &  &     4 & US (40 kHz, 55�)  &   12 �   \\ \hline
		SRF05                & 0.15 & - & 6    & \iic              &  5       &  &     4 & US (40 kHz, 55�)  &   22 �   \\ \hline
		SRF08                & 0.03 & - & 6    & \iic              &  5       &  &     3 & US (40 kHz, 55�)  &   30 �   \\ \hline
		SRF10                & 0.04 & - & 6    & Pulsweite         &  5       &  &    15 & US (40 kHz, 75�)  &   34 �   \\ \hline
		SRF235               & 0.1  & - & 1.2  & \iic              &  5       &  &       & US (235 kHz, 15�) &  100 �   \\ \hline
		URM37                & 0.04 & - & 3    & PWM, RS-323, TTL  &  5       & \< &  20 & US (55�)          &   15 \$  \\ \hline
	\end{tabular}}
	\caption{Geeignete erh�ltliche Sensoren}
	\label{tab:sensortabelle}
\end{table}
% PDF-Seite wieder auf 0� Rottation
\end{rotpdf90}
\todo{kleiner, transponiert oder separat}
