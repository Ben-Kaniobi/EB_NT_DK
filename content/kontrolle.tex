%%%%%%%%%%%%%%%%%%%%%%%%%%%%%%%%%%%%%%%%%%%%%%%%%%%%%%%%%%%%%%%%%%%%%%%%%%%%%%%
% Titel:   Kontrolle
% Autor:   Nicola K�ser
%%%%%%%%%%%%%%%%%%%%%%%%%%%%%%%%%%%%%%%%%%%%%%%%%%%%%%%%%%%%%%%%%%%%%%%%%%%%%%%
\chapter{Kontrolle}\label{ch:kontrolle}
%
	\section{Hardwaretest}\label{s:hardwaretest}
	Zum testen der Infrarotsensor-Printe wurde ein einfacher Funktionstest mit einem KO durchgef�hrt. Es funktionieren alle erstellten Printe, es ist jedoch zu erw�hnen, dass die Schrumpfschl�uche um die LEDs einen Einfluss auf die Funktion haben k�nnen. Es muss jeweils darauf geachtet werden, dass zwischen Print und Schrumpfschlauch kein Spalt ist durch den das Infrarotlicht scheinen kann. Die optimale Methode zum Anpassen der Schrumpfschl�uche ist Folgende: \SI{1}{\centi\meter} abschneiden und um einen \SI{5}{\milli\meter} Inbusschl�ssel schrumpfen. Danach kann der Schrumpfschlauch einfach auf die LEDs gesteckt werden und sollte so ideal befestigt sein.
	%
	\par Es wurde auch festgestellt, dass keine roten und blauen Schrumpfschl�uche verwendet werden k�nnen, vermutlich kann das Infrarotlicht durch diese hindurch scheinen wodurch der Sensor immer LOW ausgibt. Mit schwarzen und braunen Schrumpfschl�uchen funktioniert der Aufbau. Alle anderen Varianten wurden nicht getestet.
	\section{Softwaretest}\label{s:softwaretest}
	Damit sowohl die Adress�nderungssoftware als auch die Naherkennugssoftware sorgf�ltig getestet werden konnte wurde ein Testprotokoll erstellt und der Test damit Schritt f�r Schritt durchgef�hrt.
	Dieses Testprotokoll ist in \autoref{ch:anhang_a} ersichtlich.
