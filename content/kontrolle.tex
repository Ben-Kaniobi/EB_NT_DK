%%%%%%%%%%%%%%%%%%%%%%%%%%%%%%%%%%%%%%%%%%%%%%%%%%%%%%%%%%%%%%%%%%%%%%%%%%%%%%%
% Titel:   Kontrolle
% Autor:   Nicola K�ser
%%%%%%%%%%%%%%%%%%%%%%%%%%%%%%%%%%%%%%%%%%%%%%%%%%%%%%%%%%%%%%%%%%%%%%%%%%%%%%%
\chapter{Kontrolle}\label{ch:kontrolle}
%
	\section{Hardwaretest}\label{s:hardwaretest}
	Zum testen der Infrarotsensor-Printe wurde ein einfacher Funktionstest mit einem KO durchgef�hrt. Es funktionieren alle erstellten Printe, es ist jedoch zu erw�hnen, dass die Schrumpfschl�uche einen Einfluss auf die Funktion haben k�nnen. Es muss jeweils darauf geachtet werden, dass zwischen Print und Schrumpfschlauch kein Spalt ist durch den das Infrarotlicht scheinen kann. Die optimale Methode zum Erstellen der Schrumpfschl�uche ist folgende: \SI{1}{\centi\meter} abgeschnitten, schrumpfen um einen \SI{5}{\milli\meter} Imbussschl�ssel.
	%
	\par Es wurde auch festgestellt, dass keine roten und blauen Schrumpfschl�uche verwendet werden k�nnen, vermutlich kann das Infrarotlicht durch diese hindurch scheinen. Mit schwarzen und braunen Schrumpfschl�uchen funktioniert der Aufbau. Alle anderen Varianten wurden nicht getestet.
	\section{Softwaretest}\label{s:softwaretest}
		Testprotokoll \autoref{ch:anhang_a}
