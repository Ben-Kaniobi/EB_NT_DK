%%%%%%%%%%%%%%%%%%%%%%%%%%%%%%%%%%%%%%%%%%%%%%%%%%%%%%%%%%%%%%%%%%%%%%%%%%%%%%%
% Titel:   Schlusswort
% Autor:   Nicola K�ser
%%%%%%%%%%%%%%%%%%%%%%%%%%%%%%%%%%%%%%%%%%%%%%%%%%%%%%%%%%%%%%%%%%%%%%%%%%%%%%%
\chapter{Auswertung}\label{ch:auswertung}
\section{Ergebnis}\label{s:ergebnis}
Zur Auswertung des Ergebnisses wird das System in Bezug auf die anfangs beschriebenen Anforderungen bewertet:
\begin{description}
	\item[Genauigkeit:] Gefordert war eine Hinderniserkennung in \SI{20}{\centi\meter} Entfernung. Mit dem realisierten System ist eine Erkennung von \SI{3}{\centi\meter} bis \SI{6}{\meter} m�glich. Dieser Punkt wird also sehr gut erreicht.
	\item[Zuverl�ssigkeit:] Durch die Verwendung von zwei unterschiedlichen Sensortypen wird die Zuverl�ssigkeit stark erh�ht und entspricht somit der Anforderung.
	\item[Geschwindigkeit:] Eine Messung der Ultraschall-Module liegt aufgrund der Laufzeitmessung im Millisekundenbereich, dies scheint akzeptabel. Das Lesen der Infrarotsensoren erfolgt sehr schnell �ber GPIOs. Da die Software auf dem selben Mikrocontroller l�uft, wie das Kernsystem ist die Informations�bergabe verz�gerungsfrei m�glich.
	\item[Platzbedarf:] Die Absprache mit dem Volumenkonzept-Team wurde vor der vollst�ndigen Realisation der Hardware gemacht und es konnte eine gute Variante f�r die Montage gefunden werden.
	\item[Energieverbrauch:] Die Sensoren haben kein grosser Energieverbrauch, das System kann direkt �ber den Sensorprint versorgt werden.
	\item[Sicherheit:] Es wird kein Lasersystem oder sonstige Hardware verwendet, bei der speziell die Sicherheit f�r anwesende Personen beachtet werden muss.
\end{description}
Wie zu sehen ist wurden alle Anforderungen eingehalten, die Genauigkeit liegt sogar gut im w�nschenswerten Bereich.
%
\section{Pendenzen}\label{s:pendenzen}
In diesem Kapitel werden noch offene Punkte und m�gliche Verbesserungen erl�utert.
%
\par Was f�r den Wettbewerb noch getan werden muss, ist die Herstellung von Kabel und Printen als Ersatzmaterial und f�r den zweiten Roboter. Weiter ist die Software momentan noch ein eigenst�ndiges Projekt. Sie muss noch in die Software des Kernknotens eingebunden werden. Danach sollte das Naherkennungssystem in den Roboter eingebaut werden und so nochmals getestet werden.
%
\par Damit die Software reibungslos funktioniert, m�ssen immer beide Ultraschall-Module eingesteckt sein. Wenn dies nicht der Fall ist, liefert die \iic-Funktion zum Lesen des verbleibenden Moduls immer den gleichen Wert. Dies liegt daran, dass das Busy-Flags\footnote{Das Busy-Flag ist gesetzt w�hrend der Bus gerade f�r eine �bertragung verwendet wird.} nicht zur�ckgesetzt wird und auch nicht manuell zur�ckgesetzt werden kann (read only). Eventuell kann dieser Fehler irgendwie behoben werden, dies scheint jedoch nicht einfach durch eine Clear-Flag-Funktion realisierbar zu sein, sondern ist mit mehr Aufwand verbunden. Beispielsweise w�re es vielleicht m�glich beim Auftreten dieses Fehlers die \iic-Schnittstelle neu zu initialisieren, damit das Flag gel�scht wird und danach die Kommunikation zum nicht angeschlossenen Modul zu �berspringen.
%
\par Es wird zu diesem Zeitpunkt noch nicht �berpr�ft ob eine Messung fehlerhaft war. Damit eine Aussage dar�ber gemacht werden kann, w�re beispielsweise ein Vergleich der letzten drei Messungen m�glich. Auf diese Weises k�nnte die Wahrscheinlichkeit einer Fehlmessung stark verkleinert werden. Bei gr�sseren Fehlern der Ultraschallmessungen k�nnte auch das teurere Modul SRF235 verwendet werden. Die Ansteuerung ist gleich wie bei allen SRF-Modulen. Speziell an dieser Version ist die Ultraschallfrequenz von \SI{235}{\kilo\hertz}, wodurch der Schallkegel viel gerichteter ist.
%
\section{Schlusswort}\label{s:schlusswort}
Eurobot ist das erste Projekt dieser Gr�sse, an dem ich beteiligt bin. Diese erste Projektarbeit war interessant und eine gute Erfahrung. Trotz der vielfachen zus�tzlich aufgewendeten Zeit an Montagen und Dienstagen war es jedoch wiedermal etwas stressig, besonders gegen Ende. Nichtsdestotrotz bin ich auf die \PA{2} und besonders den Eurobot-Wettbewerb gespannt.
