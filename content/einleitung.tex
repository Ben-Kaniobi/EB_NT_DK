%%%%%%%%%%%%%%%%%%%%%%%%%%%%%%%%%%%%%%%%%%%%%%%%%%%%%%%%%%%%%%%%%%%%%%%%%%%%%%%
% Titel:   Beispiele
% Autor:   Nicola K�ser
%%%%%%%%%%%%%%%%%%%%%%%%%%%%%%%%%%%%%%%%%%%%%%%%%%%%%%%%%%%%%%%%%%%%%%%%%%%%%%%
\chapter{Einleitung}\label{ch:einleitung}

% Einf�hrung ins Thema, allgemeinverst�ndlich, Referenzen, Erl�uterung der Aufgabe

Seit 1998 findet jedes Jahr Robotik-Wettbewerb namens Eurobot\footnote{http://www.eurobot.org/} statt. Es handelt sich dabei um eine internationale Meisterschaft, in der sich Studententeams aus unterschiedlichen Schulen, oder auch Privatpersonen, in der Entwicklung eines autonomen Roboters messen k�nnen. Auch die Berner Fachhochschule nimmt seit einigen Jahren an diesem Wettbewerb teil, wobei es jeweils zuerst die nationale Meisterschaft zu bestehen gilt.
\par Die Organisatoren des Wettbewerbs geben allj�hrlich neue Spielregeln und Anforderungen an die Roboter bekannt, die Planung und das Ressourcenmanagement ist dabei den Teams �berlassen. F�r das Jahr 2014 sind zwei Roboter unterschiedlicher Gr�sse erlaubt. Die Aufgaben, mit der die Roboter Punkte sammeln k�nnen, werden jeweils unter einer zusammenfassenden Thematik ver�ffentlicht. Das aktuelle Motiv thematisiert die Urgeschichte der Menschheit und lautet "`PrehistoBot"'. Die zu l�senden Aufgaben sind die folgenden:
%
\begin{description}[leftmargin=!,labelwidth=\widthof{\bfseries Mammuts:},noitemsep]
	\item[Mammuts:] Mammutfiguren an den Spielfeldseiten mit Klebeb�llen und Netz bewerfen
	\item[Freskos:] Zeichnungen an der Spielfeldwand anbringen
	\item[Fr�chte:] Fr�chte von B�umen am Spielfeldrand pfl�cken
	\item[Feuer:] Spielkl�tze im Spielfeld sammeln und korrekt platzieren
\end{description}
%
Damit die Roboter diese Aufgaben ohne Zusammenst�sse erledigen k�nnen, m�ssen sie �ber ein System verf�gen, womit sie die n�here Umgebung um sich fortlaufend �berwachen k�nnen.
\par Dank den Teilnahmen der Berner Fachhochschule in den vergangenen Jahren, k�nnen wir teilweise auf Hardware-Komponenten und eine gewisse Erfahrung der fr�heren Teams zur�ckgreifen. Die Naherkennung war jedoch noch kein separates Teilgebiet, ausserdem sind bei den verwendeten Sensoren durch St�rungen verschiedene Fehler aufgetreten. Aus diesen Gr�nden wurde beschlossen die Naherkennung von Grund auf neu als selbst�ndiges Teilgebiet zu entwickeln. So kann diese Arbeit f�r zuk�nftige Wettbewerbe einfacher weiterverwendet, gegebenenfalls verbessert und auf die neuen Aufgaben abgestimmt werden.
\par Ziel dieser Arbeit ist also die Evaluation und Entwicklung eines Systems f�r die Naherkennung. Die Arbeit an der Naherkennung erstreckt sich �ber ein Semester und findet im Rahmen des Moduls "`BTE5511.01/02 Projektarbeit und System Engineering"' statt. Die genaue Aufgabenstellung kann dem Pflichtenheft entnommen werden.