%%%%%%%%%%%%%%%%%%%%%%%%%%%%%%%%%%%%%%%%%%%%%%%%%%%%%%%%%%%%%%%%%%%%%%%%%%%%%%%
% Titel:   Anforderungen
% Autor:   Nicola K�ser
%%%%%%%%%%%%%%%%%%%%%%%%%%%%%%%%%%%%%%%%%%%%%%%%%%%%%%%%%%%%%%%%%%%%%%%%%%%%%%%
\chapter{Umfeld und Anforderungen}\label{ch:umfeld_und_anforderungen}

%
	\section{Spielfeld}\label{s:spielfeld}
	Der Roboter wird auf einem Spielfeld der Gr�sse \SI{2}{\meter} $\cdot$ \SI{3}{\meter} eingesetzt. Die Fl�che ist von einem Rand der H�he \SI{7}{\centi\meter} umgeben. Auf dem Spielfeld gibt es diverse station�re und mobile Objekte/Hindernisse. Pro Team sind zwei Roboter erlaubt, es k�nnen sich also bis zu vier Roboter auf dem Spilfeld befinden.
	\par Die f�r die Naherkennung relevanten Aufgabenobjekte und das Spielfeld bestehen aus Holz und sind in den Farben Rot, Gelb oder Braun bemalt. Der Spielfeldrand ist in hellem grau bemalt. Zudem sind einige Objekte mit schwarzem Velcro\textsuperscript{TM} ausgestattet.
	\par Die gegnerischen Roboter k�nnen aus beliebigem Material bestehen und eine beliebige Farbe haben. Die Form der Roboter kann, innerhalb eines Umfangs von maximal \SI{1.5}{\meter}, beliebig sein.
	%
	\section{St�rungsquellen}\label{s:stoerungsquellen}
	W�hrend des Wettkampfes wird h�chstwahrscheinlich laute Musik laufen und es werden wom�glich Kameras mit Infrarot-Autofokus in Betrieb sein. Desweiteren ist zu beachten, dass die gegnerischen Roboter �ber beliebige Sensoren und Aktoren verf�gen k�nnen, was die Funktion der eigenen Sensoren beeintr�chtigen k�nnte. Auch beachtet werden muss, dass sich die Naherkennung und die eigene Navigation nicht gegenseitig st�ren darf.
	%
	\clearpage  % Neue Seite, �bersichtshalber
	%
	\section{Anforderungen}\label{s:anforderungen}
	Die Anforderungen wurden also folgendermassen definiert:
	\begin{description}
		\item[Genauigkeit:] Damit das System zuverl�ssig funktioniert, muss ein Hindernis in einer Entfernung von \SI{20}{\centi\meter} erkannt werden k�nnen, jedoch w�re eine noch bessere Erkennung w�nschenswert.
		\item[Zuverl�ssigkeit:] Da es sich bei der Naherkennung um ein Sicherheitssystem handelt, m�ssen m�gliche Fehler- und St�rungsanf�lligkeiten minimiert werden.
		\item[Geschwindigkeit:] Das System muss das Strategiesystem ohne schnittstellenbedingte Verz�gerung �ber ein Hindernis informieren k�nnen.
		\item[Platzbedarf:] Da das Volumen des Roboters relativ begrenzt ist, muss darauf geachtet werden, dass jedes Teilsystem darin unterzubringen ist. Es muss also eine Absprache mit dem Zust�ndigen Team gemacht werden.
		\item[Energieverbrauch:] Auch die Energie im Roboter ist begrenzt, da die Versorgung �ber Akkus geschiet. Desshalb gilt es den Energieverbrauch m�glichst klein zu halten.
		\item[Sicherheit:] Zur Sicherheit von anwesenden Personen, muss das System die Sicherheitsanforderungen des Eurobot-Reglements erf�llen. F�r die Naherkennung relevant sind vor allem die Bestimmungen f�r Laser. Es d�rfen nur die Laserklassen\footnote{Klassifizierung nach EN 60825-1} 1 und 1M uneingeschr�nkt verwendet werden, die Laserklasse 2 darf nur verwendet werden, wenn der Strahl zu jeder Zeit innerhalb des Spielfeldes bleibt. Alle anderen Laserklassen (namentlich Laserklassen 2M, 3R, 3B und 4) sind nicht zugelassen \cite{lit:eurobot_reglement}.
	\end{description}
