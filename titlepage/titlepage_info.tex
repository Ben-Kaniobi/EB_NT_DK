%%%%%%%%%%%%%%%%%%%%%%%%%%%%%%%%%%%%%%%%%%%%%%%%%%%%%%%%%%%%%%%%%%%%%%%%%%%%%%%
% Titel:   Beispiele
% Autor:   Nicola K�ser
%%%%%%%%%%%%%%%%%%%%%%%%%%%%%%%%%%%%%%%%%%%%%%%%%%%%%%%%%%%%%%%%%%%%%%%%%%%%%%%

In einem abteilungs\"ubergreifenden Team nimmt die Berner Fachhochschule beim allj\"ahrlich durchgef\"uhrten Eurobot Robotik-Wettbewerb teil. Ein wichtiger Punkt bei einem solchen Projekt ist immer auch die Sicherheit des Roboters. Aus diesem Grund muss die n\"ahere Umgebung um den Roboter fortlaufend \"uberwacht werden. So k\"onnen Zusammenst\"osse mit anderen Robotern und/oder anderen Objekten verhindert werden. Dieses Dokument beschreibt diese Teilaufgabe und deren Anforderungen.