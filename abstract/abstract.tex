%%%%%%%%%%%%%%%%%%%%%%%%%%%%%%%%%%%%%%%%%%%%%%%%%%%%%%%%%%%%%%%%%%%%%%%%%%%%%%%
% Titel:   Abstract
% Autor:   Nicola K�ser
%%%%%%%%%%%%%%%%%%%%%%%%%%%%%%%%%%%%%%%%%%%%%%%%%%%%%%%%%%%%%%%%%%%%%%%%%%%%%%%

%%%%%%%%%%%%%%%%%%%%%%%%%%%%%%%%%%%%%%%%%%%%%%%%%%%%%%%%%%%%%%%%%%%%%%%%%%%%%%%
\chapter*{Abstract}
In einem abteilungs�bergreifenden Team nimmt die Berner Fachhochschule bei einem allj�hrlich durchgef�hrten Robotik-Wettbewerb namens Eurobot teil. In diesem internationalen Wettbewerb k�nnen sich Studententeams aus unterschiedlichen Schulen in der Entwicklung eines autonomen Roboters messen. Die Anforderungen an die Roboter werden von den Organisatoren jedes Jahr neu aufgestellt und unter einer zusammenfassenden Thematik ver�ffentlicht. F�r das Jahr 2014 sind zwei Roboter unterschiedlicher Gr�sse erlaubt. Das Thema lautet "`PrehistoBot"' und beinhaltet Aufgaben, die die Urgeschichte der Menschheit thematisieren.
\par Damit die Roboter die Aufgaben ohne Zusammenst�sse erledigen k�nnen, m�ssen sie �ber ein System verf�gen, womit sie die n�here Umgebung um sich fortlaufend �berwachen k�nnen.
\par Daher stellte ich mir die Aufgabe, ein solches System zu entwickeln und teilte es in folgende Teilaufgaben: Recherche nach geeigneten Sensoren, Evaluation ausgew�hlter Sensoren, Entscheidung und Realisation.
\par Ich kam zum Schluss, dass die Verwendung von zwei Sensortypen, basierend auf Infrarot sowie Ultraschall, die Zuverl�ssigkeit des Systems am Besten gew�hrleistet. Das Ergebnis der Arbeit ist die Hard- und Software f�r die Naherkennung.
\todo{Evtl. offene Fragen}  % Z.B.: Noch offen ist die Art Informations�bergabe an das Kernteam, da f�r beide Roboter eine unterschiedliche...