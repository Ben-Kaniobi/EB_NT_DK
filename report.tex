%%%%%%%%%%%%%%%%%%%%%%%%%%%%%%%%%%%%%%%%%%%%%%%%%%%%%%%%%%%%%%%%%%%%%%%%%%%%%%%
% Titel:   PA1 - Dokumentation
% Autor:   Nicola K�ser
% Datum:   28.10.2013
% Version: 0.1.0
%%%%%%%%%%%%%%%%%%%%%%%%%%%%%%%%%%%%%%%%%%%%%%%%%%%%%%%%%%%%%%%%%%%%%%%%%%%%%%%
% Versionshinweise und �nderungsprotokol:
%
%  v0.0.1  2013-10-14  Erste Formulierung in Microsoft Word, mit Konzentration
%                      auf den Inhalt, da das Dokument so bald wie m�glich in
%                      LaTeX umformatiert wird.
%  v0.0.2  2013-10-23  Inhalterweiterung
%  v0.1.0  2013-10-28  Konvertierung in LaTeX
%%%%%%%%%%%%%%%%%%%%%%%%%%%%%%%%%%%%%%%%%%%%%%%%%%%%%%%%%%%%%%%%%%%%%%%%%%%%%%%
\documentclass[version=last,fleqn,numbers=noenddot]{scrreprt}
	% version=first:    Ergebniss Kompatibel zu ersten Version;
	%         last:     Ergebniss entspricht den aktuellen Paketen;
	% fleqn:            Formeln linksb�ndig;
	% numbers=noenddot: Kapitelnummerierung ohne Punkt;
	%         enddot:   Kapitelnummerierung mit Punkt;
	% twoside:          Doppelseitiger Druck

% Dokumentangaben
\newcommand{\Titel}         {Eurobot 2014 Naherkennung}
\newcommand{\Uebertitel}    {Projektarbeit 1 - Dokumentation}
\newcommand{\AutorA}        {Nicola K\"aser}
\newcommand{\Dozent}        {Prof. M. Kucera}
\newcommand{\Datum}         {\today}
\newcommand{\Ort}           {Burgdorf}
\newcommand{\Version}       {0.1.0}



%%%%%%%%%%%%%%%%%%%%%%%%%%%%%%%%%%%%%%%%%%%%%%%%%%%%%%%%%%%%%%%%%%%%%%%%%%%%%%%
% Pakete
%%%%%%%%%%%%%%%%%%%%%%%%%%%%%%%%%%%%%%%%%%%%%%%%%%%%%%%%%%%%%%%%%%%%%%%%%%%%%%%
% Maximale Tiefe des Inhaltsverzeichnis
\newcommand{\tocmaxdepth}   {2}
% Maximale Tiefe der Verzeichnis�ffnung im PDF
\newcommand{\pdfmaxdepth}   {1}
%%%%%%%%%%%%%%%%%%%%%%%%%%%%%%%%%%%%%%%%%%%%%%%%%%%%%%%%%%%%%%%%%%%%%%%%%%%%%%%
% Titel:   Bericht - Pakete
% Autor:   Simon Grossenbacher
%          Nicola K�ser (v1.1.0)
% Datum:   27.09.2013
% Version: 1.1.0
%%%%%%%%%%%%%%%%%%%%%%%%%%%%%%%%%%%%%%%%%%%%%%%%%%%%%%%%%%%%%%%%%%%%%%%%%%%%%%%
%
%:::Change-Log:::
% Versionierung erfolgt auf folgende Gegebenheiten: -1. Release Versionen
%                                                   -2. Neue Kapitel
%                                                   -3. Fehlerkorrekturen
%
% 0.0.0       Erstellung der Datei
% 1.0.0       Release
% 1.1.0       Erg�nzungen
%
%:::Hinweis:::
% Indexerstellung: makeindex -s report.ist report.idx
%   Umlaute m�ssen separat behandelt werden!
%%%%%%%%%%%%%%%%%%%%%%%%%%%%%%%%%%%%%%%%%%%%%%%%%%%%%%%%%%%%%%%%%%%%%%%%%%%%%%%

% Fix f�r KOMA-Script
\usepackage{scrhack}

% Sprach-Optionen
\usepackage[ngerman]{babel}         % Neue deutsche Rechtschreibung
\usepackage[T1]{fontenc}            % Richtige Worttrennung
%\usepackage[applemac]{inputenc}    % Mac - load extended character set (ISO 8859-1)
%\usepackage[latin1]{inputenc}      % Unix/Linux - load extended character set (ISO 8859-1)
\usepackage[ansinew]{inputenc}      % Windows - load extended character set (ISO 8859-1)
%\usepackage[utf8]{inputenc}        % UTF-8 encoding

% Zeilenabstand
\usepackage{setspace}

% Mehr Tabellenoptionen
\usepackage{tabularx}
\usepackage{longtable}

% Listen
\usepackage{enumitem}

% Besserer Flattersatz
\usepackage{ragged2e}

% Gleiten verhindern
\usepackage{float}
\usepackage{placeins}

% Z�hler per Seite resetten (z.B. Fussnoten-Index)
\usepackage{perpage}

% Ueberschriften anpassen
\usepackage{titlesec}

% Farben
\usepackage{color}
\usepackage{colortbl}  % F�r farbige Tabellen

% PDF zu Dokument hinzufuegen
\usepackage[final]{pdfpages}

% Grafiken verwalten
\usepackage{graphicx}
\usepackage[absolute]{textpos}

% Diagramme erstellen
\usepackage{pgfplots}

% Zeichnen
%\usepackage{pst-pdf}
%\usepackage{pst-all}

% Listnings verwalten
\usepackage{listings}

% Kopf- Fusszeile (Optionen m�ssen direkt �bergeben werden)
\usepackage[automark,           % Automatisches aktualisieren der Chapter-Titel
%			headsepline,        % Linie Kopfzeile
%			footsepline,        %Linie fusszeilezeile
%			markuppercase,
			plainfootsepline    % Plain-Style auch mit Linie versehen
			]{scrpage2}

% Flexible Argumente bei Funktionen
\usepackage{xargs}

% Erweiterte Steuerfunktionen
\usepackage{ifthen}
% Vereinfachungen zur ifthen Package
\newcommand{\setnewboolean}[2]{\newboolean{#1} \setboolean{#1}{#2}}
\newcommand{\ifthen}[2]{\ifthenelse{#1}{#2}{}}
\newcommand{\ifthenbool}[2]{\ifthenelse{\boolean{#1}}{#2}{}}
\newcommand{\ifthenelsebool}[3]{\ifthenelse{\boolean{#1}}{#2}{#3}}

% Erweiterte Funktionen f�rs Inhaltsverzeichnis
\usepackage{tocloft}

% Index f�r Stichwortverzeichnis
\usepackage{makeidx}

% Index f�r Literaturverzeichnis
\usepackage[babel,german=quotes]{csquotes}
%\usepackage[backend=biber,style=numeric,defernumbers=true,sorting=nyt]{biblatex}
\usepackage[backend=bibtex8,defernumbers=true]{biblatex}
\bibliography{bibliography}
\defbibheading{lit}{\section{Literatur}}
\defbibheading{pic}{\section{Abbildungen}}
\defbibheading{url}{\section{Online im Internet}}

% Zusaetzliche Symbole direkt im Text
\usepackage{textcomp}
\usepackage{amssymb}

% Einheit kontrolliert eingeben
% units -> Sch�ne Darstellung:
% Z.B. \unitfrac[1.2]{m}{s} ergibt 1.2m/s (wobei m/s wie ein Zeichen)
% siunitx -> SI-Einheiten:
% Z.B. \SI{1.2}{\meter\per\second} ergibt 1,2 ms-1
% (wobei -1 hochgestellt, L�cke vor Einheit halb so gross und Komma automatisch)
\usepackage{units}
\usepackage{siunitx}
\sisetup{locale=DE}

% Dynamische Datumsausgabe
\usepackage[german]{isodate}

% Zusaetzlich Mathemtiksymbole
\usepackage{amsmath}
\usepackage{mathtools}

% Besser Handling von internen Countern und Berechnungen
\usepackage{calc}

% ToDos anbringen am Rand
\usepackage{todonotes}
\newcommand{\ttodo}[1]{\textcolor{orange}{\textbf{#1}}}  % Text-todo

% Hyperlinks (Muss das letzte geladene Paket sein)
\usepackage[bookmarks=true,             % Verzeichnis generieren
	bookmarksopen=true,                 % Verzeichnis �ffnen
	bookmarksopenlevel=\pdfmaxdepth,    % Tiefe der Verzeichnis�ffnung
	unicode=false,                      % non-Latin Zeichen
	pdftoolbar=true,                    % PDF-Viewer Toolbar
	pdfmenubar=true,                    % PDF-Viewer Men�?
	pdffitwindow=true,                  % Fenster an Seite anpassen beim �ffnen
	pdftitle={\Titel},                  % Titel
%	pdfauthor={\Autor1, \Autor2},       % Autor
	pdfsubject={\Uebertitel},           % Thema
%	pdfcreator={\Autor1, \Autor2},      % Ersteller des Dokuments
%	pdfproducer={\Autor1, \Autor2},     % Produzent des Dokuments
	pdfnewwindow=true,                  % Links in neuem Fenster
	colorlinks=true,                    % false: Boxen-Links; true: Farben-Links
	linkcolor=black,                    % Farbe von internen Links
	citecolor=black,                    % Farbe von Links zu Bibliography
	filecolor=magenta,                  % Farbe von Links zu Dateien
	urlcolor=blue                       % Farbe von externen Links
	]{hyperref}

% Glossar/Abk�rzungsverzeichnis
\usepackage[acronym,makeindex,nowarn]{glossaries}  % nowarn: Keine Warnungen anzeigen

% �bersichtsverzeichnis
\usepackage[tight]{shorttoc}  % tight: Eintr�ge nahe zusammen (standard ist loose), funktioniert aber nicht :/




%%%%%%%%%%%%%%%%%%%%%%%%%%%%%%%%%%%%%%%%%%%%%%%%%%%%%%%%%%%%%%%%%%%%%%%%%%%%%%%
% Optionen
%%%%%%%%%%%%%%%%%%%%%%%%%%%%%%%%%%%%%%%%%%%%%%%%%%%%%%%%%%%%%%%%%%%%%%%%%%%%%%%
% Flag ob ein Abbildungs- und Tabellenverzeichnis erstellt werden soll
\setnewboolean{createLists}     {true}
% Flag ob ein Stichwortverzeichnis erstellt werden soll
\setnewboolean{createIndex}     {false}
% Flag ob ein �bersichtsverzeichnis erstellt werden soll
\setnewboolean{createShorttoc}  {false}



%%%%%%%%%%%%%%%%%%%%%%%%%%%%%%%%%%%%%%%%%%%%%%%%%%%%%%%%%%%%%%%%%%%%%%%%%%%%%%%
% Funktionen
%%%%%%%%%%%%%%%%%%%%%%%%%%%%%%%%%%%%%%%%%%%%%%%%%%%%%%%%%%%%%%%%%%%%%%%%%%%%%%%
\input{functions}



%%%%%%%%%%%%%%%%%%%%%%%%%%%%%%%%%%%%%%%%%%%%%%%%%%%%%%%%%%%%%%%%%%%%%%%%%%%%%%%
% Farben
%%%%%%%%%%%%%%%%%%%%%%%%%%%%%%%%%%%%%%%%%%%%%%%%%%%%%%%%%%%%%%%%%%%%%%%%%%%%%%%
\RequirePackage{color}                              % Color (not xcolor!)
% Allgemein
%\definecolor{grey}{gray}{0.7}
%\definecolor{lightgrey}{gray}{0.9}
%
% BFH
%\definecolor{bfhred}{rgb}{0.776,0,0.066}
%\definecolor{brickred}{cmyk}{0,0.89,0.94,0.28}      % Brickred
%\definecolor{bfhblue}{rgb}{0.396,0.49,0.56}         % Blue
\definecolor{bfhblue}{RGB}{69, 125, 151}             % Blue aus Vektorgrafik
%\definecolor{bfhorange}{rgb}{0.961,0.753,0.196}     % Orange
%\definecolor{bfhorangelight}{RGB}{246,216,136}      % Orange Light
%
% Listing
%\definecolor{hellgelb}{rgb}{1,1,0.8}
%\definecolor{listingbackground}{RGB}{246,216,136}
%\definecolor{colKeys}{rgb}{0,0,1}
%\definecolor{colIdentifier}{rgb}{0,0,0}
%\definecolor{colComments}{rgb}{0.2,0.8.3}
%\definecolor{colString}{rgb}{0,0.5,0}

% Tabellen
%\rowcolors{1}{bfhblue}{bfhorangelight}


%%%%%%%%%%%%%%%%%%%%%%%%%%%%%%%%%%%%%%%%%%%%%%%%%%%%%%%%%%%%%%%%%%%%%%%%%%%%%%%
% Schrift
%%%%%%%%%%%%%%%%%%%%%%%%%%%%%%%%%%%%%%%%%%%%%%%%%%%%%%%%%%%%%%%%%%%%%%%%%%%%%%%
% Standardschriftgr�sse
\newcommand{\defaultfontsize}{11pt}
\KOMAoptions{fontsize=\defaultfontsize}

% Zeilenabstand
%\onehalfspacing  % 1.5 Zeilenabstand

% Schrift eines bestimmten Elements anpassen/erstellen z.B. captionlabel
%\setkomafont{captionlabel}{\itshape}  % definert Schriftart f�r captionlabel
%\addkomafont{captionlabel}{\itshape}  % f�gt Eigenschaft zu captionlabel hinzu

% Formatvorlage anpassen
\setkomafont{pageheadfoot}{\footnotesize\sffamily}  % Kopf-/Fusszeile \normalfont
\setkomafont{pagenumber}{\normalfont\sffamily\bfseries}  % Seitennummer



%%%%%%%%%%%%%%%%%%%%%%%%%%%%%%%%%%%%%%%%%%%%%%%%%%%%%%%%%%%%%%%%%%%%%%%%%%%%%%%
% Gleitobjekte (Bilder/Tabellen/Formeln)
%%%%%%%%%%%%%%%%%%%%%%%%%%%%%%%%%%%%%%%%%%%%%%%%%%%%%%%%%%%%%%%%%%%%%%%%%%%%%%%
% Formelneinzug = wie Aufz�hlungseinzug
\setlength{\mathindent}{0.6\leftmargini}

% Gleitobjekte
\setlength{\intextsep}{8mm +2mm -2mm}%\intextsep10mm plus3mm minus2mm
%\setlength{\textfloatsep}{100mm plus5pt minus3pt}

% Bild-Tabellen Beschriftung linksb�ndig
%\KOMAoptions{captions=nooneline}  % linksb�ndig
\setcaphanging  % Mehrzeilige Beschriftung erh�lt Einzug



%%%%%%%%%%%%%%%%%%%%%%%%%%%%%%%%%%%%%%%%%%%%%%%%%%%%%%%%%%%%%%%%%%%%%%%%%%%%%%%
% Seiteneinstellungen
%%%%%%%%%%%%%%%%%%%%%%%%%%%%%%%%%%%%%%%%%%%%%%%%%%%%%%%%%%%%%%%%%%%%%%%%%%%%%%%
% Dokumentstadium
\KOMAoptions{draft=false}  % Erleichert das erkennen von Fehlern im Entwurfsstadium

% Papierformat
\KOMAoptions{paper=A4}  % Format (Bei direkten Massangaben: 5cm:3cm)
%\KOMAoptions{paper=landscape}  % Ausrichtung  (Standard Portrait)
\KOMAoptions{pagesize=automedia}  % Angabe f�r Ausgangstreiber

% Bindekorrektur
%\KOMAoptions{BCOR=0.1cm}  % Bindekorrektur (Bereich der durchs Binden verloren geht)

% Gr�sse des Satzspiegels
\KOMAoptions{DIV=11}  % Gr�sse des Satzspiegels (Faktor ab 4), siehe S.38 scrguide.pdf; nur f�r A4 existieren Voreinstellungen, sonst calc o. classic als Option

% Randbereich
\KOMAoptions{mpinclude=false}  % Definiert ob der Randbereich zum Textk�rper hinzugez�hlt werden soll oder nicht -> TRUE nur bei Sonderf�llen

% Doppelspaltiges Dokument
\KOMAoptions{twocolumn=false}

% Absatzabstand
\KOMAoptions{parskip=true}

% Seitenstyle
\pagestyle{scrheadings}  % myheadings scrheadings

% Platzierzung letzte Zeile
\raggedbottom  % Letze Zeile liegt dort wo sie gerade ist -> unterschiedlicher vertikaler Abstand zu Blatt Ende (unerw�nscht bei doppelseitigem Druck)
%\flushbottom  % Letze Zeile immer am Schluss -> evtl. unterschiedliche Absatzabst�nde

% Part-Seiten ohne Kopf- & Fusszeilen
\renewcommand*{\partpagestyle}{empty}

% Seitenzahl von Parts nicht im ToC anzeigen
\addtocontents{toc}{\cftpagenumbersoff{part}}

% Fussnoten-Nummerierung bei jeder Seite neu beginnen
\MakePerPage{footnote}



%%%%%%%%%%%%%%%%%%%%%%%%%%%%%%%%%%%%%%%%%%%%%%%%%%%%%%%%%%%%%%%%%%%%%%%%%%%%%%%
% Kopf-/Fusszeile
%%%%%%%%%%%%%%%%%%%%%%%%%%%%%%%%%%%%%%%%%%%%%%%%%%%%%%%%%%%%%%%%%%%%%%%%%%%%%%%
% Kopfzeile
%\clearscrheadfoot  % Alle Vorgaben l�schen (plain und scrheadings)
\renewcommand*{\chaptermarkformat}{  % headmark ohne Kapitelnummer
%\chapappifchapterprefix{\ }\thechapter\autodot\enskip
}
% i: Links; c: Zentrum; o:Rechts; [Auf Kapitelstartseiten]; {Auf allen anderen Seiten}
\ihead[]{\textcolor{bfhblue}{\headmark}}  % Chapter Titel in Kopfzeile
\chead[]{}
\ohead[\textcolor{bfhblue}{\Titel, v\Version}]{\textcolor{bfhblue}{\Titel, v\Version}}  % Titel mit Version in Kopfzeile

% Fusszeile
\ifoot[\textcolor{bfhblue}{Berner Fachhochschule}]{\textcolor{bfhblue}{Berner Fachhochschule}}  % plain und scrheadings mit Dokumenttitel versehen
\cfoot[]{}
\ofoot[\textcolor{bfhblue}{\pagemark}]{\textcolor{bfhblue}{\pagemark}}  % plain und scrheadings mit Seitennummer versehen



%%%%%%%%%%%%%%%%%%%%%%%%%%%%%%%%%%%%%%%%%%%%%%%%%%%%%%%%%%%%%%%%%%%%%%%%%%%%%%%
% Fussnote
%%%%%%%%%%%%%%%%%%%%%%%%%%%%%%%%%%%%%%%%%%%%%%%%%%%%%%%%%%%%%%%%%%%%%%%%%%%%%%%
% Fussnote
%\KOMAoptions{footnotes=multiple}  % Fussnotennummern durch "," trennen



% Satzspiegel neu berechnen -> falls andere Schriftengeladen werden und/oder Zeilenabstand ver�ndert wird
\recalctypearea



%%%%%%%%%%%%%%%%%%%%%%%%%%%%%%%%%%%%%%%%%%%%%%%%%%%%%%%%%%%%%%%%%%%%%%%%%%%%%%%
% Paket Listings Konfiguration
%%%%%%%%%%%%%%%%%%%%%%%%%%%%%%%%%%%%%%%%%%%%%%%%%%%%%%%%%%%%%%%%%%%%%%%%%%%%%%%

% XML
\lstdefinestyle{XML}{numbers=left,
	basicstyle=\scriptsize\ttfamily,
	numberstyle=\tiny,
	xleftmargin=0.5\leftmargin,
	xrightmargin=\rightmargin,
	numbersep=5pt,
	backgroundcolor=\color{listingbackground},
	breaklines=true,
	captionpos=b,
	language=XML}

% MATlAB
\lstdefinestyle{Matlab}{numbers=left,
	basicstyle=\scriptsize\ttfamily,
	numberstyle=\tiny,
	xleftmargin=0.5\leftmargin,
	xrightmargin=\rightmargin,
	numbersep=5pt,
	backgroundcolor=\color{listingbackground},
	identifierstyle=\color{colIdentifier},
	keywordstyle=\color{colKeys},
	stringstyle=\color{colString},
	commentstyle=\color{colComments},
	breaklines=true,
	captionpos=b,
	language=Matlab}

% ANSI C
\lstdefinestyle{C}{numbers=left,
	basicstyle=\scriptsize\ttfamily,
	numberstyle=\tiny,
	xleftmargin=0.5\leftmargin,
	xrightmargin=\rightmargin,
	numbersep=5pt,
	backgroundcolor=\color{listingbackground},
	identifierstyle=\color{colIdentifier},
	keywordstyle=\color{colKeys},
	stringstyle=\color{colString},
	commentstyle=\color{colComments},
	breaklines=true,
	captionpos=b,
	language=[ANSI] C}


\makeindex  % Ab hier Indexieren
%%%%%%%%%%%%%%%%%%%%%%%%%%%%%%%%%%%%%%%%%%%%%%%%%%%%%%%%%%%%%%%%%%%%%%%%%%%%%%%
% Dokument Anfang
%%%%%%%%%%%%%%%%%%%%%%%%%%%%%%%%%%%%%%%%%%%%%%%%%%%%%%%%%%%%%%%%%%%%%%%%%%%%%%%
\begin{document}
% Neuer Part im Inhaltsverzeichnis
\addcontentsline{toc}{part}{\Titel}
%
%
%%%%%%%%%%%%%%%%%%%%%%%%%%%%%%%%%%%%%%%%%%%%%%%%%%%%%%%%%%%%%%%%%%%%%%%%%%%%%%%
% Titelseite
%%%%%%%%%%%%%%%%%%%%%%%%%%%%%%%%%%%%%%%%%%%%%%%%%%%%%%%%%%%%%%%%%%%%%%%%%%%%%%%
% Seitenzahl umschalten auf eigene Nummerierung: T-1, T-2, ...
\setcounter{page}{1}
\renewcommand{\thepage}{T-\arabic{page}}
% Titelseite einf�gen
\input{titlepage/titlepage}
\thispagestyle{empty}
% Leere Seite nach Titelseite einf�gen
\cleardoublepage
%
%
%
\pagenumbering{roman}  % R�misch Nummerieren ab hier
%
%
%
%%%%%%%%%%%%%%%%%%%%%%%%%%%%%%%%%%%%%%%%%%%%%%%%%%%%%%%%%%%%%%%%%%%%%%%%%%%%%%%
% Abstract + Selbstaendige Arbeit
%%%%%%%%%%%%%%%%%%%%%%%%%%%%%%%%%%%%%%%%%%%%%%%%%%%%%%%%%%%%%%%%%%%%%%%%%%%%%%%
%%%%%%%%%%%%%%%%%%%%%%%%%%%%%%%%%%%%%%%%%%%%%%%%%%%%%%%%%%%%%%%%%%%%%%%%%%%%%%%
% Titel:   Abstract
% Autor:   Nicola K�ser
%%%%%%%%%%%%%%%%%%%%%%%%%%%%%%%%%%%%%%%%%%%%%%%%%%%%%%%%%%%%%%%%%%%%%%%%%%%%%%%

%%%%%%%%%%%%%%%%%%%%%%%%%%%%%%%%%%%%%%%%%%%%%%%%%%%%%%%%%%%%%%%%%%%%%%%%%%%%%%%
\chapter*{Abstract}

\chapter*{Selbständige Arbeit}
\label{ch:Selbständige Arbeit}
Ich erkläre ausdrückich, dass es sich bei dieser von mir eingereichten Arbeit um eine von mir selbst und ohne unerlaubte Beihilfe sowie in eigenen Worten verfasste Originalarbeit handelt.
Ich bestätige überdies, dass die Arbeit als Ganze oder in Teilen weder bereits einmal zur Abgeltung anderer Studienleistungen an der Berner Fachhochschule oder an einer anderen Universität oder Ausbildungseinrichtung eingereicht worden ist noch inskünftig durch mein Zutun als Abgeltung einer weiteren Studienleistung eingereicht werden wird.
Ich erkläre ausdrücklich, dass ich sämtliche in der oben genannten Arbeit enthaltenen Bezüge auf fremde Quellen als solche kenntlich gemacht haben.
\vspace{2cm}
\begin{tabbing}
xxxxxxxxxxxxxxxxxxxx\=xxxxxxxxxxxxxxxxxxxxxxx \kill
Ort, Datum      \>  \Ort, \Datum \\ \\ \\

Vorname Name    \> \AutorA \\  \\ 
Unterschrift    \> ......................................................... \\ \\  \\ 
\end{tabbing}

\clearpage
%
%
%
%%%%%%%%%%%%%%%%%%%%%%%%%%%%%%%%%%%%%%%%%%%%%%%%%%%%%%%%%%%%%%%%%%%%%%%%%%%%%%%
% �bersichts- & Inhaltsverzeichnis
%%%%%%%%%%%%%%%%%%%%%%%%%%%%%%%%%%%%%%%%%%%%%%%%%%%%%%%%%%%%%%%%%%%%%%%%%%%%%%%
% �bersichtsverzeichnis
\ifthenbool{createShorttoc}{
	% �bersicht mit definierter Tiefe
	\shorttoc{�bersicht}{0}
}
%
\clearpage
%
% Inhaltsverzeichnis
%\KOMAoptions{toc=listof}   % Abbildungs- und Tabellenverzeichnis ins Inhaltsverzeichnis
\KOMAoptions{toc=index}     % Stichwortverzeichnis ins Inhaltsverzeichnis
%
% Tiefe der Nummerierung
\setcounter{secnumdepth}{3}
% Tiefe der Auflistung im Inhaltsverzeichnis
\setcounter{tocdepth}{\tocmaxdepth}
%
% Inhaltsverzeichnis linksb�ndig
%\KOMAoptions{toc=flat}
%
% Verzeichnisse mit einer Kapitelnummer versehen
%\KOMAoptions{toc=listofnumbered}
%
% Inhaltsverzeichnis einf�gen
\tableofcontents
%
\clearpage  % Seite beenden
%
%
%
%%%%%%%%%%%%%%%%%%%%%%%%%%%%%%%%%%%%%%%%%%%%%%%%%%%%%%%%%%%%%%%%%%%%%%%%%%%%%%%
% Dokumentinhalt
%%%%%%%%%%%%%%%%%%%%%%%%%%%%%%%%%%%%%%%%%%%%%%%%%%%%%%%%%%%%%%%%%%%%%%%%%%%%%%%
% Dokument arabisch Nummerieren
\pagenumbering{arabic}
%
% Kapitel 0: Beispiele
%\input{content/0_beispiele}
%
% Einleitung
%%%%%%%%%%%%%%%%%%%%%%%%%%%%%%%%%%%%%%%%%%%%%%%%%%%%%%%%%%%%%%%%%%%%%%%%%%%%%%%
% Titel:   Beispiele
% Autor:   Nicola K�ser
%%%%%%%%%%%%%%%%%%%%%%%%%%%%%%%%%%%%%%%%%%%%%%%%%%%%%%%%%%%%%%%%%%%%%%%%%%%%%%%
\chapter{Einleitung}\label{ch:einleitung}

% Einf�hrung ins Thema, allgemeinverst�ndlich, Referenzen, Erl�uterung der Aufgabe

Seit 1998 findet jedes Jahr Robotik-Wettbewerb namens Eurobot\footnote{\url{http://www.eurobot.org/}} statt. Es handelt sich dabei um eine internationale Meisterschaft, in der sich Studententeams aus unterschiedlichen Schulen, oder auch Privatpersonen, in der Entwicklung eines autonomen Roboters messen k�nnen. Auch die Berner Fachhochschule nimmt seit einigen Jahren an diesem Wettbewerb teil, wobei es jeweils zuerst die nationale Meisterschaft zu bestehen gilt.
\par Die Organisatoren des Wettbewerbs geben allj�hrlich neue Spielregeln und Anforderungen an die Roboter bekannt, die Planung und das Ressourcenmanagement ist dabei den Teams �berlassen. F�r das Jahr 2014 sind zwei Roboter unterschiedlicher Gr�sse erlaubt. Die Aufgaben, mit der die Roboter Punkte sammeln k�nnen, werden jeweils unter einer zusammenfassenden Thematik ver�ffentlicht. Das aktuelle Motiv thematisiert die Urgeschichte der Menschheit und lautet "`PrehistoBot"'. Die zu l�senden Aufgaben sind die folgenden:
%
\begin{description}[leftmargin=!,labelwidth=\widthof{\bfseries Mammuts:},noitemsep]
	\item[Mammuts:] Mammutfiguren an den Spielfeldseiten mit Klebeb�llen und Netz bewerfen
	\item[Freskos:] Zeichnungen an der Spielfeldwand anbringen
	\item[Fr�chte:] Fr�chte von B�umen am Spielfeldrand pfl�cken
	\item[Feuer:] Spielkl�tze im Spielfeld sammeln und korrekt platzieren
\end{description}
%
Damit die Roboter diese Aufgaben ohne Zusammenst�sse erledigen k�nnen, m�ssen sie �ber ein System verf�gen, womit sie die n�here Umgebung um sich fortlaufend �berwachen k�nnen.
\par Dank den Teilnahmen der Berner Fachhochschule in den vergangenen Jahren, k�nnen wir teilweise auf Hardware-Komponenten und eine gewisse Erfahrung der fr�heren Teams zur�ckgreifen. Die Naherkennung war jedoch noch kein separates Teilgebiet, ausserdem sind bei den verwendeten Sensoren durch St�rungen verschiedene Fehler aufgetreten. Aus diesen Gr�nden wurde beschlossen die Naherkennung von Grund auf neu als selbst�ndiges Teilgebiet zu entwickeln. So kann diese Arbeit f�r zuk�nftige Wettbewerbe einfacher weiterverwendet, gegebenenfalls verbessert und auf die neuen Aufgaben abgestimmt werden.
\par Ziel dieser Arbeit ist also die Evaluation und Entwicklung eines Systems f�r die Naherkennung. Die Arbeit an der Naherkennung erstreckt sich �ber ein Semester und findet im Rahmen des Moduls "`BTE5511.01/02 Projektarbeit und System Engineering"' statt. Die genaue Aufgabenstellung kann dem Pflichtenheft entnommen werden.
%
% Anforderungen
%%%%%%%%%%%%%%%%%%%%%%%%%%%%%%%%%%%%%%%%%%%%%%%%%%%%%%%%%%%%%%%%%%%%%%%%%%%%%%%
% Titel:   Anforderungen
% Autor:   Nicola K�ser
%%%%%%%%%%%%%%%%%%%%%%%%%%%%%%%%%%%%%%%%%%%%%%%%%%%%%%%%%%%%%%%%%%%%%%%%%%%%%%%
\chapter{Umfeld und Anforderungen}\label{ch:umfeld-und-anforderungen}

%
	\section{Spielfeld}\label{s:spielfeld}
	Der Roboter wird auf einem Spielfeld der Gr�sse 3 m x 2 m eingesetzt. Die Fl�che ist von einem Rand der H�he 7 cm umgeben. Auf dem Spielfeld gibt es diverse station�re und mobile Objekte/Hindernisse. Pro Team sind zwei Roboter erlaubt, es k�nnen sich also bis zu vier Roboter auf dem Spilfeld befinden.
	\par Die f�r die Naherkennung relevanten Aufgabenobjekte und das Spielfeld bestehen aus Holz und sind in den Farben Rot, Gelb oder Braun bemalt. Der Spielfeldrand ist in hellem grau bemalt. Zudem sind einige Objekte mit schwarzem Velcro\textsuperscript{TM} ausgestattet.
	\par Die gegnerischen Roboter k�nnen aus beliebigem Material bestehen und eine beliebige Farbe haben. Die Form der Roboter kann, innerhalb eines Umfangs von maximal 1500 mm, beliebig sein.
	%
	\section{St�rungsquellen}\label{s:stoerungsquellen}
	W�hrend des Wettkampfes wird h�chstwahrscheinlich Musik laufen und es werden wom�glich Kameras mit IR-Autofokus in Betrieb sein. Desweiteren ist zu beachten, dass die gegnerischen Roboter �ber beliebige Sensoren und Aktoren verf�gen kann, die die Funktion der eigenen Sensoren beeintr�chtigen kann. Auch beachtet werden muss, dass sich die Naherkennung und die eigene Navigation nicht gegenseitig st�ren darf.
	%
	\clearpage  % Neue Seite, �bersichtshalber
	%
	\section{Anforderungen}\label{s:anforderungen}
	Die Anforderungen wurden also folgendermassen definiert:
	\begin{description}
		\item[Genauigkeit:] Damit das System zuverl�ssig funktioniert, muss ein Hindernis in einer Entfernung von 20 cm erkannt werden k�nnen, jedoch w�re eine noch bessere Erkennung w�nschenswert.
		\item[Zuverl�ssigkeit:] Da es sich bei der Naherkennung um ein Sicherheitssystem handelt, m�ssen m�gliche Fehler- und St�rungsanf�lligkeiten minimiert werden.
		\item[Geschwindigkeit:] Das System muss das Strategiesystem ohne schnittstellenbedingte Verz�gerung �ber ein Hindernis informieren k�nnen.
		\item[Platzbedarf:] Da das Volumen des Roboters relativ begrenzt ist, muss darauf geachtet werden, dass jedes Teilsystem darin unterzubringen ist. Es muss also eine Absprache mit dem Zust�ndigen Team gemacht werden.
		\item[Energieverbrauch:] Auch die Energie in einem mobilen System wie ein Roboter ist begrenzt. Desshalb gilt es den Energieverbrauch m�glichst klein zu halten.
		\item[Sicherheit:] Zur Sicherheit von anwesenden Personen, muss das System die Sicherheitsanforderungen des Eurobot-Reglements erf�llen. F�r die Naherkennung relevant sind vor allem die Bestimmungen f�r Laser. Es d�rfen nur die Laserklassen\footnote{Klassifizierung nach EN 60825-1} 1 und 1M uneingeschr�nkt verwendet werden, die Laserklasse 2 darf nur verwendet werden, wenn der Strahl zu jeder Zeit innerhalb des Spielfeldes bleibt. Alle anderen Laserklassen (namentlich Laserklassen 2M, 3R, 3B und 4) sind nicht zugelassen. \cite{lit:eurobot_reglement}
	\end{description}
%
% Recherche
%%%%%%%%%%%%%%%%%%%%%%%%%%%%%%%%%%%%%%%%%%%%%%%%%%%%%%%%%%%%%%%%%%%%%%%%%%%%%%%
% Titel:   Recherche
% Autor:   Nicola K�ser
%%%%%%%%%%%%%%%%%%%%%%%%%%%%%%%%%%%%%%%%%%%%%%%%%%%%%%%%%%%%%%%%%%%%%%%%%%%%%%%
\chapter{Recherche}\label{ch:recherche}
Zu Beginn mussten Informationen dar�ber beschafft werden, mit welchen Methoden eine Distanzmessung realisiert werden kann. Diese konnten dann in zwei Kategorien eingeordnet werden, eine Kategorie die sich f�r den Eurobot-Wettbewerb eignet und eine Kategorie die aus verschiedenen Gr�nden weniger gut geeignet ist.
%
	\section{Ungeeignete Sensormethoden}\label{s:ungeeignete_sensormethoden}
	In diese Kategorie geh�ren Kontaktsensoren, da die Naherkennung dazu dienen soll den Kontakt mit dem Gegner zu verhindern. Weiter dazu geh�ren Kamerasysteme, da f�r die Verarbeitung der Daten sehr viel Systemleistung aufgewendet werden muss, und eine gute Software mehr Zeit und Aufwand erfordert als f�r diese Arbeit geplant ist. Auch Induktive- und kapazitive Sensoren eignen sich f�r die Naherkennung nicht, da diese nur Distanzen im Millimeterbereich messen.
	\par Ein Radarsysteme, wie sie beispielsweise f�r automatische T�ren verwendet werden, w�re theoretisch gut geeignet. Besonders die Tatsache, dass damit auch die Geschwindigkeit des Gegnerroboters ermittelt werden k�nnte w�re von Vorteil. Es ist jedoch so, dass die erh�ltlichen Radarsensoren jeweils eine Reichweite erst ab ca. zwei Metern aufweisen.
	%
	\section{Geeignete Sensormethoden}\label{s:geeignete_sensormethoden}
	Die geeigneten Sensortypen kann man wiederum in drei Unterkategorien mit unterschiedlichen Vor- und Nachteilen unterordnen.
	%
		\subsection{Infrarot}\label{ss:infrarort}
		Die meisten Infrarot-Sensoren basieren auf dem Prinzip der Triangulation. Eine Lichtquelle im Sensor, meistens eine LED, strahlt Infrarotlicht aus. Trifft das Licht auf ein Objekt, so wird es  in einem zur Distanz im Verh�ltnis stehenden Winkel zur�ckgeworfen. Beim Sensor trifft dieses Licht auf ein Array von Infrarot-Empf�ngern, wodurch der Winkel und somit auch die Distanz zum Objekt berechnet werden kann.
		\image{content/image/ir_prinzip}{scale=.3}[Prinzip der Infrarot-Distanzmessung \cite{pic:ir}][Prinzip der Infrarot-Distanzmessung][pic:ir]
		\par Die fr�heren Eurobot-Teilnehmer der Berner Fachhochschule verwendeten haupts�chlich Infrarotsensoren von Sharp\footnote{\url{http://www.sharp.ch/}} die dieses Prinzip verwenden. Sie stellten damit jedoch einige Probleme fest, so kam es w�hrend dem Wettbewerb vermehrt zu Aussetzern und Fehlmessungen \cite{lit:gegnerischer_roboter}. Diese Fehlmessungen wurden vermutlich wegen St�rungen durch den Autofokus von Fotoapparaten hervorgerufen. Mit moduliertem Infrarotlicht k�nnten solche St�rungen beseitigt werden. M�glicherweise ist dies jedoch mit einem Array von Infrarot-Empf�ngern schwieriger zu realisieren, denn die Sensoren die mit einer Modulation arbeiten greifen nicht auf die Triangulation zur�ck.
		\par So ist eine weitere M�glichkeit die bei Infrarot angewendet wird die Distanzermittlung mittels der Lichtintensit�t. Die Sensoren die auf diesem Prinzip basieren messen aber f�r gew�hnlich nicht direkt die Distanz, sondern haben einfach eine bestimmte Schaltschwelle und liefern somit also einfach zwei verschiedene Zust�nde.
		\par Mit Infrarot w�re auch noch eine Messung per Laufzeitverfahren m�glich, dazu wurden aber keine Sensoren gefunden.
		%
		\subsection{Ultraschall}\label{ss:ultraschall}
		Da Schall viel langsamer ist als Licht, wird die Laufzeitmessung mit Ultraschall h�ufiger eingesetzt. Ein Ultraschallgeber sendet ein Signal aus, welches dann von einem Objekt reflektiert wird. Mit einem Ultraschallsensor kann das Echo dann wieder empfangen werden. Mit der Zeitdifferenz und der Schallgeschwindigkeit kann dann die Distanz berechnet werden:
		\formula{
			s=\frac{1}{2}\cdot c_{Schall} \cdot \Delta t
		}{
			$s$ & Distanz / \SI{}{\meter}\\
			$c\low{Schall}$ & Schallgeschwindigkeit / \SI{}{\meter\per\second} ($\approx~\SI{343.2}{\meter\per\second}$)\\
			\Delta $t$ & Zeitdifferenz / \SI{}{\second}
		}[eq:laufzeitmessung]
		%
		\par Je nach Ultraschallfrequenz variiert die Gr�sse des Messkegels. Der Vorteil von Ultraschallsensoren im Gegensatz zu Infrarotsensoren ist, dass das Material und die Farbe des zu detektierenden Objekts keinen Einfluss auf die Messung hat. In der Thesisarbeit hat ein Vorg�ngerteam beim Evaluieren verschiedener Sensoren Probleme mit bewegten Objekten aufgrund des Dopplereffektes\footnote{Dopplereffekt: Durch zeitliche Stauchung/Dehnung eines Signals bei Abstands�nderungen zwischen Sender und Empf�nger wird die Frequenz ver�ndert wahrgenommen.} festgestellt \cite{lit:gegnerischer_roboter}.
		%
		\subsection{Laser}\label{ss:laser}
		F�r m�glichst genaue Messungen verwendet man am besten Laserensoren. Die meisten Lasersensoren nutzen auch die Triangulation, im Gegensatz zu Sensoren mit Infrarot-LEDs erh�lt man aber aufgrund der Lasereigenschaften eine viel genauere Messung. Ein Nachteil liegt auf der Hand: Durch die genauere Messung ist der Messpunkt auch sehr viel kleiner. M�chte man einen gr�sseren Bereich abdecken, m�ssen desshalb entweder mehrere Laser verwendet werden, oder es muss mittels mechanischer Konstruktion mehrmals an verschiedenen Stellen gemessen werden.
		%
	\section{Sensorliste}\label{ss:sensorliste}
	Mit der oben genannten Kategorisierung wurde eine Liste von geeigneten erh�ltlichen Sensoren zusammengestellt:
	%%%%%%%%%%%%%%%%%%%%%%%%%%%%%%%%%%%%%%%%%%%%%%%%%%%%%%%%%%%%%%%%%%%%%%%%%%%%%%%
% Titel:   Sensortabelle
% Autor:   Nicola K�ser
%%%%%%%%%%%%%%%%%%%%%%%%%%%%%%%%%%%%%%%%%%%%%%%%%%%%%%%%%%%%%%%%%%%%%%%%%%%%%%%
\begin{table}[htbp]
	% Lokale Commands (nur g�ltig bis \end{table})
	\newcommand{\T}[1]{\textcolor{white}{#1}}  % Titel
	\newcommand{\n}[2][c]{\begin{tabular}[#1]{@{}c@{}}#2\end{tabular}}  % Spezielle Zelle in der Umruch (\\) verwendet werden kann
	\centering
	\rotatebox{90}{
	\begin{tabular}{|l|l|l|l|l|l|l|}
		\hline \rowcolor{bfhblue}
		\T{Bezeichnung}            & \T{Messbereich} & \T{Output}        & \T{Vcc}      & \T{I (typ.)} & \T{Methode}   & \T{Preis (ca.)} \\ \hline
		GP2Y0A02                   & 0.2 m - 1.5 m   & 2.6 V - 0.4 V     & 5 V          & 33 mA    & IR                & 20 �    \\ \hline
		GP2Y0A21                   & 0.1 m - 0.8 m   & 2.6 V - 0.4 V     & 5 V          & 30 mA    & IR                & 14 �    \\ \hline
		IS471F                     & je nach LED     & Schaltschwelle    & 4.5 V - 16 V & 3.5 mA   & IR                & 2.5 �   \\ \hline
		SRF10                      & 0.04 m - 6 m    & Pulsweite         & 5 V          & 15 mA    & US (40 kHz, 75�)  & 46 �    \\ \hline
		SRF02                      & 0.01 m - 4 m    & iic \& RS-232     & 5 V          & 4 mA     & US (40 kHz, 55�)  & 23 �    \\ \hline
		SRF05                      & 0.15 m - 6 m    & iic               & 5 V          & 4 mA     & US (40 kHz, 55�)  & 22 �    \\ \hline
		GP2Y0A41                   & 0.04 m - 0.3 m  & 2.7 V - 0.4 V     & 5 V          & 12 mA    & IR                & 20 CHF  \\ \hline
		GP2D02                     & 0.1 m - 0.8 m   & Seriell 8-Bit     & 4.4 V - 7 V  & 22 mA    & IR                & ~       \\ \hline
		SRF08                      & 0.03 m - 6 m    & iic               & 5 V          & 3 mA     & US (40 kHz,55�)   & 26 �    \\ \hline
		URM37                      & 0.04 m - 3 m    & PWM, RS-323, TTL  & 5 V          & < 20 mA  & US (55�)          & 15 \$   \\ \hline
		\n{VDM28-8-L\\-IO/37c/122} & 0.2 m - 8 m     & IO-Link           & 18 V - 30 V  & ~        & Laser (Cl. 2)     & 530 CHF \\ \hline
		\n{VDM28-8-L\\-IO/37c/136} & 0.2 m - 8 m     & IO-Link           & 18 V - 30 V  & ~        & Laser (Cl. 2)     & 510 CHF \\ \hline
		O1D102                     & 0.2 m - 3.5 m   & 2 Schaltschwellen & 18 V - 30 V  & < 150 mA & Laser (Cl. 2)     & 640 CHF \\ \hline
		O1D100                     & 0.2 m - 10 m    & 2 Schaltschwellen & 18 V - 30 V  & < 150 mA & Laser (Cl. 2)     & 640 CHF \\ \hline
		\n{OADM\\13U7480/S35A}     & 0.05 m - 0.55 m & 0 V - 10 V        & 12 V - 28 V  & < 80 mA  & Laser (Cl. 2)     & 780 �   \\ \hline
		\n{Hokuyo\\URG-04LX}       & 0.06 m - 4.1 m  & USB 2.0, RS-232   & 5 V          & 500 mA   & Laser (Cl. 1)     & 1600 �  \\ \hline
		\n{FHDK\\10P1101/KS35}     & 0.2 m - 1.2 m   & Schaltschwelle    & 10 V - 30 V  & 20 mA    & IR                & 130 \$  \\ \hline
		\n{FHCK\\07P6901/KS35A}    & 0.1 m - 0.6 m   & Schaltschwelle    & 10 V - 30 V  & 20 mA    & IR                & 160 �   \\ \hline
		SRF235                     & 0.1 m - 1.2 m   & iic               & 5 V          & ~        & US (235 kHz, 15�) & 100 �   \\ \hline
		SEN13635B                  & 0.03 m - 4 m    & iic               & 5 V          & 15 mA    & US (40 kHz, 30�)  & 12 \$   \\ \hline
	\end{tabular}}
	\caption{Tabelle mit m�glichen Sensoren}
	\label{tab:sensortabelle}
\end{table}

test


%
% Evaluation und Entscheidung
%%%%%%%%%%%%%%%%%%%%%%%%%%%%%%%%%%%%%%%%%%%%%%%%%%%%%%%%%%%%%%%%%%%%%%%%%%%%%%%
% Titel:   Evaluation
% Autor:   Nicola K�ser
%%%%%%%%%%%%%%%%%%%%%%%%%%%%%%%%%%%%%%%%%%%%%%%%%%%%%%%%%%%%%%%%%%%%%%%%%%%%%%%
\chapter{Evaluation}\label{ch:evaluation}
Mit Hilfe der Liste konnte entschieden werden, welche Sensoren f�r die Evaluation bestellt werden. Einerseits wurden diverse Infrarot-Sensoren ausgew�hlt, anderseits auch Ultraschall-Module. Ein Laser-Sensor der in einer Eurobot-Thesis verwendet wurde, wurde auch noch dazu genommen. So konnte ein Einblick in alle erl�uterten  geeigneten Methoden gewonnen werden und die Vor- und Nachteile abgewogen werden.
%
	\section{GP2D120 und GP2Y0A02YK}\label{s:gp2d120-und-gp2y0a02yk}
	Die Infrarot-Sensoren GP2D120 und GP2Y0A02YK von Sharp basieren auf dem Triangulationsprinzip. Es sind Nachfolgemodelle der Typen wie die fr�heren Teams sie verwendeten.
	\begin{figure}[H]%H htbp
		\centering
		\begin{tikzpicture}
	\begin{axis}[
		height=9cm, width=13cm,
		xlabel=Distanz/m, ylabel=U/V,
		grid=major,
		legend style={cells={anchor=west}}]
	%
	%\addplot[color=blue,mark=*] coordinates {
	\addplot coordinates {
		( 10, 2.32)
		( 12, 2.7 )
		( 15, 2.73)
		( 18, 2.6 )
		( 20, 2.52)
		( 30, 1.94)
		( 40, 1.48)
		( 50, 1.18)
		( 60, 0.98)
		( 80, 0.73)
		(100, 0.59)
		(140, 0.37)
	};
	\addlegendentry{Holz}
	%
	%\addplot[color=red,mark=square*] coordinates {
	\addplot coordinates {
		( 10, 2.73)
		( 12, 2.8 )
		( 15, 2.57)
		( 18, 2.3 )
		( 20, 2.2 )
		( 30, 1.73)
		( 40, 1.44)
		( 50, 1.14)
		( 60, 1.03)
		( 80, 0.79)
		(100, 0.66)
		(140, 0.5 )
	};
	\addlegendentry{Metall, gl�nzend}
	%
	%\addplot[color=teal,mark=triangle*] coordinates {
	\addplot coordinates {
		( 10, 2.52)
		( 12, 2.8 )
		( 15, 2.78)
		( 18, 2.62)
		( 20, 2.52)
		( 30, 1.96)
		( 40, 1.5 )
		( 50, 1.2 )
		( 60, 1   )
		( 80, 0.75)
		(100, 0.58)
		(140, 0.4 )
	};
	\addlegendentry{Metall, matt}
	%
	\end{axis}
\end{tikzpicture}
		\caption[Messungen mit dem IR-Sensor GP2D120]{Messungen mit dem Infrarotsensor GP2D120}
		\label{img:messung-gp2d120}
	\end{figure}
	\begin{figure}[H]%H htbp
		\centering
		\begin{tikzpicture}
	\begin{axis}[
		height=9cm, width=13cm,
		xlabel=Distanz/m, ylabel=U/V,
		grid=major,
		legend style={cells={anchor=west}}]
	%
	%\addplot[color=blue,mark=*] coordinates {
	\addplot coordinates {
		( 1  , 2.05)
		( 1.5, 1.86)
		( 2  , 2.65)
		( 2.5, 3.06)
		( 2.7, 3.06)
		( 3  , 2.96)
		( 3.5, 2.82)
		( 4  , 2.54)
		( 5  , 2.13)
		( 6  , 1.85)
		( 7  , 1.62)
		( 8  , 1.44)
		(10  , 1.14)
		(15  , 0.77)
		(20  , 0.57)
		(30  , 0.34)
		(40  , 0.25)
	};
	\addlegendentry{Holz}
	%
	%\addplot[color=red,mark=square*] coordinates {
	\addplot coordinates {
		( 1  , 2.76 )
		( 1.5, 2.64 )
		( 2  , 2.56 )
		( 2.5, 2.57 )
		( 2.7, 2.54 )
		( 3  , 2.44 )
		( 3.5, 2.327)
		( 4  , 2.11 )
		( 5  , 1.84 )
		( 6  , 1.56 )
		( 7  , 1.39 )
		( 8  , 1.28 )
		(10  , 1.05 )
		(15  , 0.74 )
		(20  , 0.57 )
		(30  , 0.44 )
		(40  , 0.34 )
	};
	\addlegendentry{Metall, gl�nzend}
	%
	%\addplot[color=teal,mark=triangle*] coordinates {
	\addplot coordinates {
		( 1  , 2.69)
		( 1.5, 2.22)
		( 2  , 2.67)
		( 2.5, 3.02)
		( 2.7, 3.06)
		( 3  , 3.01)
		( 3.5, 2.83)
		( 4  , 2.51)
		( 5  , 2.13)
		( 6  , 1.85)
		( 7  , 1.63)
		( 8  , 1.47)
		(10  , 1.2 )
		(15  , 0.81)
		(20  , 0.61)
		(30  , 0.37)
		(40  , 0.27)
	};
	\addlegendentry{Metall, matt}
	%
	\end{axis}
\end{tikzpicture}
		\caption[Messungen mit dem IR-Sensor GP2Y0A02YK]{Messungen mit dem Infrarotsensor GP2Y0A02YK}
		\label{img:messung-gp2y0a02yk}
	\end{figure}
	Die im vorherigen Kapitel beschriebenen Probleme wie Aussetzer und Messfehler konnten nicht reproduziert werden. Es gab einige Abweichungen wegen spiegelnder Oberfl�che der Objekte, dieses Ph�nomen ist aber gut nachvollziehbar. Im Diagramm ist jedoch zu erkennen, dass auch bei diesen Sensor-Versionen die in der Thesis beschriebene Problematik der zweideutigen Messresultate \cite{lit:gegnerischer_roboter}  besteht.
	\section{IS471F}\label{s:is471f}
	Auch von Sharp, ist der Sensor IS471F. Dabei handelt es sich um ein Bauteil, das in einer einfachen Schaltung als Distanzschalter funktioniert.
	\image{content/image/is471f}{scale=.5}[Einfache Beispielschaltung mit Sensor IS471F \cite{pic:is471f}][Einfache Beispielschaltung mit Sensor IS471F][pic:is471f]
	Die im Sensor integrierte Schaltung steuert externe Infrarot-LEDs an. Ist das so modulierte Licht f�r den Sensor sichtbar, wo liefert dieser einen definierten Pegel, andernfalls den invertierten Pegel. Dies funktioniert nicht nur bei direkter Bestrahlung, wie bei Lichtschranken, sondern auch bei Reflektiertem Licht. F�r die Naherkennung k�nnte also �ber die Lichtst�rke die Schaltdistanz eingestellt werden.
	\begin{figure}[H]%H htbp
		\centering
		%%%%%%%%%%%%%%%%%%%%%%%%%%%%%%%%%%%%%%%%%%%%%%%%%%%%%%%%%%%%%%%%%%%%%%%%%%%%%%%
% Titel:   Diagramm: IS471F
% Autor:   Nicola K�ser
%%%%%%%%%%%%%%%%%%%%%%%%%%%%%%%%%%%%%%%%%%%%%%%%%%%%%%%%%%%%%%%%%%%%%%%%%%%%%%%
\begin{tikzpicture}
	\begin{axis}[
		ybar,
		bar width=5mm,
		height=4.095cm, width=9cm,
		ylabel=Distanz/m,
		ymajorgrids=true,
		enlargelimits=0.5,
		legend style={
			cells={anchor=west},
			legend pos=outer north east
		},
		xtick=data,
		nodes near coords,
		nodes near coords align={vertical},
		symbolic x coords={Normale LED, Helle LED}]
	%
	\addplot coordinates{(Normale LED, 17) (Helle LED, 35)};
	\addplot coordinates{(Normale LED, 30) (Helle LED, 47)};
	\addplot coordinates{(Normale LED, 40) (Helle LED, 52)};
	\addplot coordinates{(Normale LED, 27) (Helle LED, 37)};
	\addplot coordinates{(Normale LED,  6) (Helle LED,  9)};
	\legend{Holz, {Metall, matt}, {Metall, gl�nzend}, {Metall, eloxiert schwarz}, {Plastik, matt schwarz}}
	%
	\end{axis}
\end{tikzpicture}
		\caption[Messungen mit dem IR-Sensor IS471F]{Messungen mit dem Infrarotsensor IS471F}
		\label{img:messung-is471f}
	\end{figure}
	\todo{auswertung diagramm}
	%
	\section{SRF-Module von Devantech}\label{s:srf-module-von-devantech}
	Devantech\footnote{\url{http://www.devantech.co.uk/}} entwickelt diverse elektronische Module, darunter befindet sich auch eine Reihe von Ultraschall-Modulen zur Distanzmessung.
	Die \ttodo{blablabla} bla von 
		\subsection{SRF02}
		\subsection{SRF08}
		\subsection{SRF10}
	\section{Lasersensor von Baumer}\label{s:lasersensor-von-baumer}
	Der Sensor OADM 13U7480/S35A von Baumer\footnote{\url{http://www.baumer.com/ch-de/}}
%
% Realisation
%%%%%%%%%%%%%%%%%%%%%%%%%%%%%%%%%%%%%%%%%%%%%%%%%%%%%%%%%%%%%%%%%%%%%%%%%%%%%%%
% Titel:   Realisation
% Autor:   Nicola K�ser
%%%%%%%%%%%%%%%%%%%%%%%%%%%%%%%%%%%%%%%%%%%%%%%%%%%%%%%%%%%%%%%%%%%%%%%%%%%%%%%
\chapter{Realisation}\label{ch:realisation}
%
	\section{Hardware}\label{s:hardware}
	Es wird je ein Ultraschallmodul vorne und eines hinten am Roboter montiert. Die Infrarotsensoren werden auch vorne und hinten, je nach Platz positioniert.
	Die Ultraschallmodule m�ssen nur an den \iic-Bus angeschlossen werden und ben�tigen keine weitere Hardware. F�r die Infrarotsensoren ist jedoch eine einfache Schaltung notwendig. Die zwei LEDs werden �ber ein Trimm-Potentiometer am Sensor angeschlossen, so kann gegebenenfalls die Schaltdistanz herabgesetzt werden.
	\image{content/image/ir-schema}{scale=.4}[Schema f�r Infrarotsensor][Schema f�r Infrarotsensor][pic:schema]
	Da die Schaltung nicht besonders aufwendig ist, sind die Printe als Veroboard-Aufbau realisiert. Auf Wunsch des Volumenkonzept-Teams wurden vorerst zwei Varianten des Printlayouts erstellt, eine quadratische- und eine schmale Version. Da der Platz des kleinen Roboters relativ knapp ist, konnte so die bestm�gliche Option ausgew�hlt werden.
	\begin{figure}[H]
	\centering
	\begin{minipage}[t]{.49\textwidth}
		\centering
		\includegraphics[scale=.19]{content/image/hw_quadratisch}
		\captionof{figure}[Hardwarevariante 1 - Quadratisch]{Variante 1 - Quadratisch}
		\label{pic:hardvarevariante1}
	\end{minipage}
	\begin{minipage}[t]{.49\textwidth}
		\centering
		\includegraphics[scale=.19]{content/image/hw_schmal}
		\captionof{figure}[Hardwarevariante 2 - Schmal]{Variante 2 - Schmal}
		\label{pic:hardvarevariante2}
	\end{minipage}
	\end{figure}
	%
	Schlussendlich wurde zusammen die Entscheidung gef�llt beide Varianten zu verwenden. Folgende Konstellation erwies sich als passend:
	
	\begin{description}[leftmargin=!,labelwidth=\widthof{\bfseries Quadratische Version:},noitemsep]
		\item[Schmale Version:] Vorne mittig
		\item[Quadratische Version:] Hinten mittig, vorne rechts und links
	\end{description}
	%
	\section{Software}\label{s:software}
	F�r die Software wird das Echtzeitbetriebssystem FreeRTOS\footnote{Free real-time operating system: \url{http://www.freeRTOS.org/}} verwendet. Beim kleineren Roboter l�uft die Naherkennungs-Software auf dem RoboBoard zusammen mit weiteren Teilgebieten.
	%
		\subsection{Set\_SRF08\_address}\label{ss:set_srf08_address}
		Damit am gleichen \iic-Bus beide Ultraschallsensoren verwendet werden k�nnen, musste zuerst eine separate Software zum �ndern der Slave-Adresse erstellt werden. Der Ablauf der folgende:
		\begin{enumerate}
			\item �berpr�ffen ob neue Adresse g�ltig ist. Falls nicht, Programm nicht fortsetzten.
			\item Erste Pr�fsequenz zur Adress�nderung via \iic\ an Kommandoregister senden und definierte Zeit warten.
			\item Zweite Pr�fsequenz zur Adress�nderung via \iic\ an Kommandoregister senden und definierte Zeit warten.
			\item Dritte Pr�fsequenz zur Adress�nderung via \iic\ an Kommandoregister senden und definierte Zeit warten.
			\item Neue Adresse via \iic\ an Kommandoregister senden.
			\item Beim Aus- und erneuten Einstecken sollte nun die LED auf dem Modul mit dem zur neuen Adresse geh�renden Code (siehe Spezifikationen \cite{url:spec_srf08}) blinken.
		\end{enumerate}
		%
		Der vordere Sensor behielt seine Standard-Adresse (\hex{E0}), dem hinteren wurde die Adresse \hex{E2} neu zugewiesen.
		%
		\subsection{Rangefinder}\label{ss:rangefinder}
		F�r die Naherkennung wurden zwei Task implementiert, einer f�r die Infrarot- und einer f�r die Ultraschallsensoren.
		%
			\subsubsection{IR-Task}\label{sss:ir-task}
			Da ein Infrarotsensor lediglich einen GPIO\footnote{General purpose input/output} ben�tigt, ist der Taskablauf leicht zusammengefasst:
			\begin{enumerate}
				\item GPIOs initialisieren.
				\item Endlosschlaufe des folgenden Ablaufs:
				\begin{enumerate}
					\item Alle vier GPIOs lesen.
					\item �berpr�fen ob einer davon ein Hindernis erkennt.
					\begin{enumerate}
						\item Falls ein Hindernis erkannt ist, IR-Alarmflag setzen.
						\item Andernfalls IR-Alarmflag zur�cksetzen.
					\end{enumerate}
					\item Definierte Zeit warten, so dass der Task immer im gleichen Takt ausgef�hrt wird.
				\end{enumerate}
			\end{enumerate}
			%
			\subsubsection{US-Task}\label{sss:us-task}
			Der Task f�r die Ultraschallsensoren ist etwas komplizierter, wesshalb die Darstellung als Flussdiagram besser geeignet ist. F�r den \iic-Zugriff wurde das Mutex-Verfahren\footnote{Mit dem Mutex-Verfahren kann der gleichzeitige Zugriff mehrerer Tasks auf die selbe Schnittstelle verhindert werden.} verwendet, zur �berschaubarkeit zeit das Diagramm diesen Teil jedoch nicht.
			\image{content/image/flussdiagramm_us-task}{scale=1}[Flussdiagramm des US-Tasks][Flussdiagramm des US-Tasks][pic:flussdiagramm_us-task]
			\clearpage
			%
				\paragraph{Funktion setSRF08Range und setSRF08Gain}\label{par:setsrf08range_setsrf08gain}
				Diese Funktionen dienen dem einmaligen Einstellen der maximalen Reichweite und Verst�rkung. 
				%
				\par F�r die \textbf{Reichweite} wird zuerst der �bergebene Wert auf den zul�ssigen Bereich angepasst. Anschliessend wird der Registerwert mit folgender Formel berechnet und per \iic\ an das Modul geschickt. F�r dieses Projekt wird momentan der Standardwert verwendet.
				\formula{
					x&=\frac{s - \SI{43}{\milli\meter}}{\SI{43}{\milli\meter}}
				}{
					$x$ & Registerwert (\SI{1}{Byte})\\
					$s$ & Reichweite in Millimeter
				}[eq:gain]
				Als maximale Reichweite wird in dieser Version \SI{1}{\meter} verwendet. Im Header-File\footnote{"`src/application/Rangefinder.h"'} ist daf�r ein Makro\footnote{"`RANGEFINDER\_RANGE"'} definiert mit dem der Werte schnell angepasst werden kann.
				%
				\par Der Registerwert der \textbf{Verst�rkung} kann durch Probieren auf die Anwendung angepasst werden. In der Funktion wird wie bei der Reichweite zuerst der Wert �berpr�ft, gegebenenfalls angepasst und dann an das Modul geschickt.
				%
				\paragraph{Funktion startSRF08Meas und readSRF08Meas}\label{par:startsrf08meas_readsrf08meas} Diese Funktionen dienen zur Durchf�hrung einer Messung.
				%
				\par Mit der Funktion \textbf{startSRF08Meas} wird das Kommando zum Starten der Messung in Millimeter per \iic\ an das Modul geschickt. Danach muss eine gewisse Zeit gewartet werden, da das Modul die Echos abwartet und in dieser Zeit nicht verwendet werden kann. F�r das Modul SRF08 mit Standardverst�rkung ist diese Zeit \SI{65}{\milli\second}.\par Nach dem Warten liefert die Funktion \textbf{readSRF08Meas} die gemessene Distanz in Millimeter. Das Modul sendet den Wert in zwei Byte, diese m�ssen also noch zusammengef�gt werden.
				\image{content/image/flussdiagramm_readsrf08meas}{scale = 1, trim = 0 25px 0 25px, clip}[Flussdiagramm der Funktion readSRF08Meas][Flussdiagramm der Funktion readSRF08Meas][pic:flussdiagramm_readsrf08meas]
		%

%
% Kontrolle
%%%%%%%%%%%%%%%%%%%%%%%%%%%%%%%%%%%%%%%%%%%%%%%%%%%%%%%%%%%%%%%%%%%%%%%%%%%%%%%
% Titel:   Kontrolle
% Autor:   Nicola K�ser
%%%%%%%%%%%%%%%%%%%%%%%%%%%%%%%%%%%%%%%%%%%%%%%%%%%%%%%%%%%%%%%%%%%%%%%%%%%%%%%
\chapter{Kontrolle}\label{ch:kontrolle}
%
	\section{Hardwaretest}\label{s:hardwaretest}
	Zum testen der Infrarotsensor-Printe wurde ein einfacher Funktionstest mit einem KO durchgef�hrt. Es funktionieren alle erstellten Printe, es ist jedoch zu erw�hnen, dass die Schrumpfschl�uche einen Einfluss auf die Funktion haben k�nnen. Es muss jeweils darauf geachtet werden, dass zwischen Print und Schrumpfschlauch kein Spalt ist durch den das Infrarotlicht scheinen kann. Die optimale Methode zum Erstellen der Schrumpfschl�uche ist folgende: \SI{1}{\centi\meter} abgeschnitten, schrumpfen um einen \SI{5}{\milli\meter} Imbussschl�ssel.
	%
	\par Es wurde auch festgestellt, dass keine roten und blauen Schrumpfschl�uche verwendet werden k�nnen, vermutlich kann das Infrarotlicht durch diese hindurch scheinen. Mit schwarzen und braunen Schrumpfschl�uchen funktioniert der Aufbau. Alle anderen Varianten wurden nicht getestet.
	\section{Softwaretest}\label{s:softwaretest}
		Testprotokoll \autoref{ch:anhang_a}

%
% Schlusskapitel
%%%%%%%%%%%%%%%%%%%%%%%%%%%%%%%%%%%%%%%%%%%%%%%%%%%%%%%%%%%%%%%%%%%%%%%%%%%%%%%
% Titel:   Schlusswort
% Autor:   Nicola K�ser
%%%%%%%%%%%%%%%%%%%%%%%%%%%%%%%%%%%%%%%%%%%%%%%%%%%%%%%%%%%%%%%%%%%%%%%%%%%%%%%
\chapter{Auswertung}\label{ch:auswertung}
\section{Ergebnis}\label{s:ergebnis}
Zur Auswertung des Ergebnisses wird das System in Bezug auf die anfangs beschriebenen Anforderungen bewertet:
\begin{description}
	\item[Genauigkeit:] Gefordert war eine Hinderniserkennung in \SI{20}{\centi\meter} Entfernung. Mit dem realisierten System ist eine Erkennung von \SI{3}{\centi\meter} bis \SI{6}{\meter} m�glich. Dieser Punkt wird also sehr gut erreicht.
	\item[Zuverl�ssigkeit:] Durch die Verwendung von zwei unterschiedlichen Sensortypen wird die Zuverl�ssigkeit stark erh�ht und entspricht somit der Anforderung.
	\item[Geschwindigkeit:] Eine Messung der Ultraschall-Module liegt aufgrund der Laufzeitmessung im Millisekundenbereich, dies scheint akzeptabel. Das Lesen der Infrarotsensoren erfolgt sehr schnell �ber GPIOs. Da die Software auf dem selben Mikrocontroller l�uft, wie das Kernsystem ist die Informations�bergabe verz�gerungsfrei m�glich.
	\item[Platzbedarf:] Die Absprache mit dem Volumenkonzept-Team wurde vor der vollst�ndigen Realisation der Hardware gemacht und es konnte eine gute Variante f�r die Montage gefunden werden.
	\item[Energieverbrauch:] Die Sensoren haben kein grosser Energieverbrauch, das System kann direkt �ber den Sensorprint versorgt werden.
	\item[Sicherheit:] Es wird kein Lasersystem oder sonstige Hardware verwendet, bei der speziell die Sicherheit f�r anwesende Personen beachtet werden muss.
\end{description}
Wie zu sehen ist wurden alle Anforderungen eingehalten, die Genauigkeit liegt sogar gut im w�nschenswerten Bereich.
%
\section{Pendenzen}\label{s:pendenzen}
In diesem Kapitel werden noch offene Punkte und m�gliche Verbesserungen erl�utert.
%
\par Was f�r den Wettbewerb noch getan werden muss, ist die Herstellung von Kabel und Printen als Ersatzmaterial und f�r den zweiten Roboter. Weiter ist die Software momentan noch ein eigenst�ndiges Projekt. Sie muss noch in die Software des Kernknotens eingebunden werden. Danach sollte das Naherkennungssystem in den Roboter eingebaut werden und so nochmals getestet werden.
%
\par Damit die Software reibungslos funktioniert, m�ssen immer beide Ultraschall-Module eingesteckt sein. Wenn dies nicht der Fall ist, liefert die \iic-Funktion zum Lesen des verbleibenden Moduls immer den gleichen Wert. Dies liegt daran, dass das Busy-Flags\footnote{Das Busy-Flag ist gesetzt w�hrend der Bus gerade f�r eine �bertragung verwendet wird.} nicht zur�ckgesetzt wird und auch nicht manuell zur�ckgesetzt werden kann (read only). Eventuell kann dieser Fehler irgendwie behoben werden, dies scheint jedoch nicht einfach durch eine Clear-Flag-Funktion realisierbar zu sein, sondern ist mit mehr Aufwand verbunden. Beispielsweise w�re es vielleicht m�glich beim Auftreten dieses Fehlers die \iic-Schnittstelle neu zu initialisieren, damit das Flag gel�scht wird und danach die Kommunikation zum nicht angeschlossenen Modul zu �berspringen.
%
\par Es wird zu diesem Zeitpunkt noch nicht �berpr�ft ob eine Messung fehlerhaft war. Damit eine Aussage dar�ber gemacht werden kann, w�re beispielsweise ein Vergleich der letzten drei Messungen m�glich. Auf diese Weises k�nnte die Wahrscheinlichkeit einer Fehlmessung stark verkleinert werden. Bei gr�sseren Fehlern der Ultraschallmessungen k�nnte auch das teurere Modul SRF235 verwendet werden. Die Ansteuerung ist gleich wie bei allen SRF-Modulen. Speziell an dieser Version ist die Ultraschallfrequenz von \SI{235}{\kilo\hertz}, wodurch der Schallkegel viel gerichteter ist.
%
\section{Schlusswort}\label{s:schlusswort}
Eurobot ist das erste Projekt dieser Gr�sse, an dem ich beteiligt bin. Diese erste Projektarbeit war interessant und eine gute Erfahrung. Trotz der vielfachen zus�tzlich aufgewendeten Zeit an Montagen und Dienstagen war es jedoch wiedermal etwas stressig, besonders gegen Ende. Nichtsdestotrotz bin ich auf die Projektarbeit 2 und besonders den Eurobot-Wettbewerb gespannt.

%
%
%
%%%%%%%%%%%%%%%%%%%%%%%%%%%%%%%%%%%%%%%%%%%%%%%%%%%%%%%%%%%%%%%%%%%%%%%%%%%%%%%
% Hilfs�bersichten
%%%%%%%%%%%%%%%%%%%%%%%%%%%%%%%%%%%%%%%%%%%%%%%%%%%%%%%%%%%%%%%%%%%%%%%%%%%%%%%
\KOMAoptions{listof=leveldown}  % Verzeichnisse als Section statt Kapitel
%
% List of figures as Section
%\renewcommand\listoffigures{%
	%\section{\listfigurename}% Used to be \chapter{\listfigurename}
		%%\@mkboth{\MakeUppercase\listfigurename}%
			%%{\MakeUppercase\listfigurename}%
	%\@starttoc{lof}%
%}
%
% List of tables as Section
%\renewcommand\listoftables{%
	%\section{\listtablename}% Used to be \chapter{\listfigurename}
		%\@mkboth{\MakeUppercase\listtablename}%
			%{\MakeUppercase\listtablename}%
	%\@starttoc{lot}%
%}
%
\ifthenbool{createLists}{
	% Kapitel ohne Nummerierung, aber in Inhaltsverzeichnis
	\chapternn{Abbildungs- und Tabellen�bersicht}
	%
	% Abbildungsverzeichnis
	\listoffigures
	\todo{section not chapter}
	%
	% Tabellenverzeichnis
	\listoftables
	\todo{section not chapter}
}
%
%
%
%%%%%%%%%%%%%%%%%%%%%%%%%%%%%%%%%%%%%%%%%%%%%%%%%%%%%%%%%%%%%%%%%%%%%%%%%%%%%%%
% Quellen
%%%%%%%%%%%%%%%%%%%%%%%%%%%%%%%%%%%%%%%%%%%%%%%%%%%%%%%%%%%%%%%%%%%%%%%%%%%%%%%
% �berpr�fen ob mindestens 1x \cite verwendet wurde, sonst Kapitel weglassen
\ifnum\value{numcite}>0
	% Kapitel ohne Nummerierung, aber in Inhaltsverzeichnis
	\chapternn{Quellen}
	In diesem Abschnitt sind alle Quellen des Dokumentes verzeichnet. Wird keine Referenz auf eine Quelle angegeben, so handelt es sich um selbst erarbeitete Inhalte des Autors.
	%
	% Literatur
	\printbibliography[heading=lit,keyword=lit,prefixnumbers=Lit,resetnumbers=true]
	%
	% Abbildungen
	\printbibliography[heading=pic,keyword=pic,prefixnumbers=Abb,resetnumbers=true]
	%
	% Online
	\printbibliography[heading=url,keyword=url,prefixnumbers=URL,resetnumbers=true]
\fi
%
%
%
%%%%%%%%%%%%%%%%%%%%%%%%%%%%%%%%%%%%%%%%%%%%%%%%%%%%%%%%%%%%%%%%%%%%%%%%%%%%%%%
% Stichwortverzeichnis
%%%%%%%%%%%%%%%%%%%%%%%%%%%%%%%%%%%%%%%%%%%%%%%%%%%%%%%%%%%%%%%%%%%%%%%%%%%%%%%
\ifthenbool{createIndex}{
	\renewcommand{\indexname}{Stichwortverzeichnis}
	\printindex
}
%
%
%
%%%%%%%%%%%%%%%%%%%%%%%%%%%%%%%%%%%%%%%%%%%%%%%%%%%%%%%%%%%%%%%%%%%%%%%%%%%%%%%
% Anhang
%%%%%%%%%%%%%%%%%%%%%%%%%%%%%%%%%%%%%%%%%%%%%%%%%%%%%%%%%%%%%%%%%%%%%%%%%%%%%%%
% Neue Seite, damit Seitennummerierung richtig.
% (Ohne wird das lezte Kapitel schon neu nummeriert, und wenn \pagenumbering
% nach \part verwendet wird stimmt die Seitenzahl vom Part-Titel nicht.)
\clearpage
%
% Anhang im Inhaltsverzeichnis zu unterst auf der Seite anzeigen
%\addtocontents{toc}{\vfill}
%
% Seitenzahl umschalten auf alphabetisches Nummerieren (nur f�r Titelblatt)
\pagenumbering{Alph}
%
% Titel zur Abgrenzung des Anhangs, ohne Nummerierung im Inhaltsverzeichnis
\partnn{Anhang}\label{p:anhang}
%
% Auf Anhang umschalten
\appendix
%
% Anhang einf�gen
%%%%%%%%%%%%%%%%%%%%%%%%%%%%%%%%%%%%%%%%%%%%%%%%%%%%%%%%%%%%%%%%%%%%%%%%%%%%%%%
% Titel:   Anhang
% Autor:   Nicola K�ser
%%%%%%%%%%%%%%%%%%%%%%%%%%%%%%%%%%%%%%%%%%%%%%%%%%%%%%%%%%%%%%%%%%%%%%%%%%%%%%%
% Umschalten der Seitennummerierung auf (Kapitelbuchstabe)-(Seite, 0-indexiert)
\newcommand{\mypagenumbering}{
	% Seitenzahl umschalten auf eigene Nummerierung: A-1, A-2, ...
	\renewcommand{\thepage}{\Alph{chapter}-\arabic{page}}
	% Seitenz�hler auf 0, damit die Seite 1 des Anhangs dann auch A-1
	\setcounter{page}{0}
}
%
% Anhang A
\chapter{Testprotokoll v1.0}\label{ch:anhang_a}
\mypagenumbering
\includepdf[scale=1,pages=-]{appendix/anhangA/Testprotokoll_v1_0}
%
% Anhang B
\chapter{Testprotokoll v1.1}\label{ch:anhang_b}
\mypagenumbering
\includepdf[scale=1,pages=-]{appendix/anhangB/Testprotokoll_v1_1}
%
% Anhang C
\chapter{Testprotokoll v1.2}\label{ch:anhang_c}
\mypagenumbering
\includepdf[scale=1,pages=-]{appendix/anhangC/Testprotokoll_v1_2}
%
%
%
\end{document}