%%%%%%%%%%%%%%%%%%%%%%%%%%%%%%%%%%%%%%%%%%%%%%%%%%%%%%%%%%%%%%%%%%%%%%%%%%%%%%%
% Titel:   PA1 - Dokumentation
% Autor:   Nicola K�ser
% Datum:   28.10.2013
% Version: 0.1.0
%%%%%%%%%%%%%%%%%%%%%%%%%%%%%%%%%%%%%%%%%%%%%%%%%%%%%%%%%%%%%%%%%%%%%%%%%%%%%%%
% Versionshinweise und �nderungsprotokol:
%
%  v0.0.1  2013-10-14  Erste Formulierung in Microsoft Word, mit Konzentration
%                      auf den Inhalt, da das Dokument so bald wie m�glich in
%                      LaTeX umformatiert wird.
%  v0.0.2  2013-10-23  Inhalterweiterung
%  v0.1.0  2013-10-28  Konvertierung in LaTeX
%%%%%%%%%%%%%%%%%%%%%%%%%%%%%%%%%%%%%%%%%%%%%%%%%%%%%%%%%%%%%%%%%%%%%%%%%%%%%%%
\documentclass[version=last,fleqn,numbers=noenddot]{scrreprt}
	% version=first:    Ergebniss Kompatibel zu ersten Version;
	%         last:     Ergebniss entspricht den aktuellen Paketen;
	% fleqn:            Formeln linksb�ndig;
	% numbers=noenddot: Kapitelnummerierung ohne Punkt;
	%         enddot:   Kapitelnummerierung mit Punkt;
	% twoside:          Doppelseitiger Druck

% Dokumentangaben
\newcommand{\Titel}         {Eurobot 2014 Naherkennung}
\newcommand{\Uebertitel}    {Projektarbeit 1 - Dokumentation}
\newcommand{\AutorA}        {Nicola K\"aser}
\newcommand{\Dozent}        {M. Kucera}
\newcommand{\Datum}         {\today}
\newcommand{\Ort}           {Burgdorf}
\newcommand{\Version}       {0.1.0}



%%%%%%%%%%%%%%%%%%%%%%%%%%%%%%%%%%%%%%%%%%%%%%%%%%%%%%%%%%%%%%%%%%%%%%%%%%%%%%%
% Pakete
%%%%%%%%%%%%%%%%%%%%%%%%%%%%%%%%%%%%%%%%%%%%%%%%%%%%%%%%%%%%%%%%%%%%%%%%%%%%%%%
% Maximale Tiefe des Inhaltsverzeichnis und Tiefe der Verzeichnis�ffnung im PDF
\newcommand{\tocmaxdepth}   {2}
%%%%%%%%%%%%%%%%%%%%%%%%%%%%%%%%%%%%%%%%%%%%%%%%%%%%%%%%%%%%%%%%%%%%%%%%%%%%%%%
% Titel:   Bericht - Pakete
% Autor:   Simon Grossenbacher
%          Nicola K�ser (v1.1.0)
% Datum:   27.09.2013
% Version: 1.1.0
%%%%%%%%%%%%%%%%%%%%%%%%%%%%%%%%%%%%%%%%%%%%%%%%%%%%%%%%%%%%%%%%%%%%%%%%%%%%%%%
%
%:::Change-Log:::
% Versionierung erfolgt auf folgende Gegebenheiten: -1. Release Versionen
%                                                   -2. Neue Kapitel
%                                                   -3. Fehlerkorrekturen
%
% 0.0.0       Erstellung der Datei
% 1.0.0       Release
% 1.1.0       Erg�nzungen
%
%:::Hinweis:::
% Indexerstellung: makeindex -s report.ist report.idx
%   Umlaute m�ssen separat behandelt werden!
%%%%%%%%%%%%%%%%%%%%%%%%%%%%%%%%%%%%%%%%%%%%%%%%%%%%%%%%%%%%%%%%%%%%%%%%%%%%%%%

% Fix f�r KOMA-Script
\usepackage{scrhack}

% Sprach-Optionen
\usepackage[ngerman]{babel}         % Neue deutsche Rechtschreibung
\usepackage[T1]{fontenc}            % Richtige Worttrennung
%\usepackage[applemac]{inputenc}    % Mac - load extended character set (ISO 8859-1)
%\usepackage[latin1]{inputenc}      % Unix/Linux - load extended character set (ISO 8859-1)
\usepackage[ansinew]{inputenc}      % Windows - load extended character set (ISO 8859-1)
%\usepackage[utf8]{inputenc}        % UTF-8 encoding

% Zeilenabstand
\usepackage{setspace}

% Mehr Tabellenoptionen
\usepackage{tabularx}
\usepackage{longtable}

% Listen
\usepackage{enumitem}

% Besserer Flattersatz
\usepackage{ragged2e}

% Gleiten verhindern
\usepackage{float}
\usepackage{placeins}

% Z�hler per Seite resetten (z.B. Fussnoten-Index)
\usepackage{perpage}

% Ueberschriften anpassen
\usepackage{titlesec}

% Farben
\usepackage{color}
\usepackage{colortbl}  % F�r farbige Tabellen

% PDF zu Dokument hinzufuegen
\usepackage[final]{pdfpages}

% Grafiken verwalten
\usepackage{graphicx}
\usepackage[absolute]{textpos}

% Diagramme erstellen
\usepackage{pgfplots}

% Zeichnen
%\usepackage{pst-pdf}
%\usepackage{pst-all}

% Listnings verwalten
\usepackage{listings}

% Kopf- Fusszeile (Optionen m�ssen direkt �bergeben werden)
\usepackage[automark,           % Automatisches aktualisieren der Chapter-Titel
%			headsepline,        % Linie Kopfzeile
%			footsepline,        %Linie fusszeilezeile
%			markuppercase,
			plainfootsepline    % Plain-Style auch mit Linie versehen
			]{scrpage2}

% Flexible Argumente bei Funktionen
\usepackage{xargs}

% Erweiterte Steuerfunktionen
\usepackage{ifthen}
% Vereinfachungen zur ifthen Package
\newcommand{\setnewboolean}[2]{\newboolean{#1} \setboolean{#1}{#2}}
\newcommand{\ifthen}[2]{\ifthenelse{#1}{#2}{}}
\newcommand{\ifthenbool}[2]{\ifthenelse{\boolean{#1}}{#2}{}}
\newcommand{\ifthenelsebool}[3]{\ifthenelse{\boolean{#1}}{#2}{#3}}

% Erweiterte Funktionen f�rs Inhaltsverzeichnis
\usepackage{tocloft}

% Index f�r Stichwortverzeichnis
\usepackage{makeidx}

% Index f�r Literaturverzeichnis
\usepackage[babel,german=quotes]{csquotes}
%\usepackage[backend=biber,style=numeric,defernumbers=true,sorting=nyt]{biblatex}
\usepackage[backend=bibtex8,defernumbers=true]{biblatex}
\bibliography{bibliography}
\defbibheading{lit}{\section{Literatur}}
\defbibheading{pic}{\section{Abbildungen}}
\defbibheading{url}{\section{Online im Internet}}

% Zusaetzliche Symbole direkt im Text
\usepackage{textcomp}
\usepackage{amssymb}

% Einheit kontrolliert eingeben
% units -> Sch�ne Darstellung:
% Z.B. \unitfrac[1.2]{m}{s} ergibt 1.2m/s (wobei m/s wie ein Zeichen)
% siunitx -> SI-Einheiten:
% Z.B. \SI{1.2}{\meter\per\second} ergibt 1,2 ms-1
% (wobei -1 hochgestellt, L�cke vor Einheit halb so gross und Komma automatisch)
\usepackage{units}
\usepackage{siunitx}
\sisetup{locale=DE}

% Dynamische Datumsausgabe
\usepackage[german]{isodate}

% Zusaetzlich Mathemtiksymbole
\usepackage{amsmath}
\usepackage{mathtools}

% Besser Handling von internen Countern und Berechnungen
\usepackage{calc}

% ToDos anbringen am Rand
\usepackage{todonotes}
\newcommand{\ttodo}[1]{\textcolor{orange}{\textbf{#1}}}  % Text-todo

% Hyperlinks (Muss das letzte geladene Paket sein)
\usepackage[bookmarks=true,             % Verzeichnis generieren
	bookmarksopen=true,                 % Verzeichnis �ffnen
	bookmarksopenlevel=\pdfmaxdepth,    % Tiefe der Verzeichnis�ffnung
	unicode=false,                      % non-Latin Zeichen
	pdftoolbar=true,                    % PDF-Viewer Toolbar
	pdfmenubar=true,                    % PDF-Viewer Men�?
	pdffitwindow=true,                  % Fenster an Seite anpassen beim �ffnen
	pdftitle={\Titel},                  % Titel
%	pdfauthor={\Autor1, \Autor2},       % Autor
	pdfsubject={\Uebertitel},           % Thema
%	pdfcreator={\Autor1, \Autor2},      % Ersteller des Dokuments
%	pdfproducer={\Autor1, \Autor2},     % Produzent des Dokuments
	pdfnewwindow=true,                  % Links in neuem Fenster
	colorlinks=true,                    % false: Boxen-Links; true: Farben-Links
	linkcolor=black,                    % Farbe von internen Links
	citecolor=black,                    % Farbe von Links zu Bibliography
	filecolor=magenta,                  % Farbe von Links zu Dateien
	urlcolor=blue                       % Farbe von externen Links
	]{hyperref}

% Glossar/Abk�rzungsverzeichnis
\usepackage[acronym,makeindex,nowarn]{glossaries}  % nowarn: Keine Warnungen anzeigen

% �bersichtsverzeichnis
\usepackage[tight]{shorttoc}  % tight: Eintr�ge nahe zusammen (standard ist loose), funktioniert aber nicht :/




%%%%%%%%%%%%%%%%%%%%%%%%%%%%%%%%%%%%%%%%%%%%%%%%%%%%%%%%%%%%%%%%%%%%%%%%%%%%%%%
% Optionen
%%%%%%%%%%%%%%%%%%%%%%%%%%%%%%%%%%%%%%%%%%%%%%%%%%%%%%%%%%%%%%%%%%%%%%%%%%%%%%%
% Flag ob ein Abbildungs- und Tabellenverzeichnis erstellt werden soll
\setnewboolean{createLists}     {true}
% Flag ob ein Stichwortverzeichnis erstellt werden soll
\setnewboolean{createIndex}     {true}
% Flag ob ein �bersichtsverzeichnis erstellt werden soll
\setnewboolean{createShorttoc}  {true}



%%%%%%%%%%%%%%%%%%%%%%%%%%%%%%%%%%%%%%%%%%%%%%%%%%%%%%%%%%%%%%%%%%%%%%%%%%%%%%%
% Funktionen
%%%%%%%%%%%%%%%%%%%%%%%%%%%%%%%%%%%%%%%%%%%%%%%%%%%%%%%%%%%%%%%%%%%%%%%%%%%%%%%
\input{functions}



%%%%%%%%%%%%%%%%%%%%%%%%%%%%%%%%%%%%%%%%%%%%%%%%%%%%%%%%%%%%%%%%%%%%%%%%%%%%%%%
% Farben
%%%%%%%%%%%%%%%%%%%%%%%%%%%%%%%%%%%%%%%%%%%%%%%%%%%%%%%%%%%%%%%%%%%%%%%%%%%%%%%
\RequirePackage{color}                              % Color (not xcolor!)
% Allgemein
\definecolor{grey}{gray}{0.7}
\definecolor{lightgrey}{gray}{0.9}

% BFH
\definecolor{bfhred}{rgb}{0.776,0,0.066}
\definecolor{brickred}{cmyk}{0,0.89,0.94,0.28}      % Brickred
\definecolor{bfhblue}{rgb}{0.396,0.49,0.56}         % Blue
\definecolor{bfhorange}{rgb}{0.961,0.753,0.196}     % Orange
\definecolor{bfhorangelight}{RGB}{246,216,136}      % Orange Light

% Listing
\definecolor{hellgelb}{rgb}{1,1,0.8}
\definecolor{listingbackground}{RGB}{246,216,136}
\definecolor{colKeys}{rgb}{0,0,1}
\definecolor{colIdentifier}{rgb}{0,0,0}
\definecolor{colComments}{rgb}{0.2,0.8.3}
\definecolor{colString}{rgb}{0,0.5,0}

% Tabellen
%\rowcolors{1}{bfhblue}{bfhorangelight}


%%%%%%%%%%%%%%%%%%%%%%%%%%%%%%%%%%%%%%%%%%%%%%%%%%%%%%%%%%%%%%%%%%%%%%%%%%%%%%%
% Schrift
%%%%%%%%%%%%%%%%%%%%%%%%%%%%%%%%%%%%%%%%%%%%%%%%%%%%%%%%%%%%%%%%%%%%%%%%%%%%%%%
% Standardschriftgr�sse
\KOMAoptions{fontsize=11pt}

% Zeilenabstand
%\onehalfspacing  % 1.5 Zeilenabstand

% Schrift eines bestimmten Elements anpassen/erstellen z.B. captionlabel
%\setkomafont{captionlabel}{\itshape}  % definert Schriftart f�r captionlabel
%\addkomafont{captionlabel}{\itshape}  % f�gt Eigenschaft zu captionlabel hinzu

% Formatvorlage anpassen
\setkomafont{pageheadfoot}{\footnotesize\sffamily}  % Kopf-/Fusszeile \normalfont
\setkomafont{pagenumber}{\normalfont\sffamily\bfseries}  % Seitennummer



%%%%%%%%%%%%%%%%%%%%%%%%%%%%%%%%%%%%%%%%%%%%%%%%%%%%%%%%%%%%%%%%%%%%%%%%%%%%%%%
% Gleitobjekte (Bilder/Tabellen/Formeln)
%%%%%%%%%%%%%%%%%%%%%%%%%%%%%%%%%%%%%%%%%%%%%%%%%%%%%%%%%%%%%%%%%%%%%%%%%%%%%%%
% Formelneinzug = wie Aufz�hlungseinzug
\setlength{\mathindent}{0.6\leftmargini}

% Gleitobjekte
\setlength{\intextsep}{8mm +2mm -2mm}%\intextsep10mm plus3mm minus2mm
%\setlength{\textfloatsep}{100mm plus5pt minus3pt}

% Bild-Tabellen Beschriftung linksb�ndig
%\KOMAoptions{captions=nooneline}  % linksb�ndig
\setcaphanging  % Mehrzeilige Beschriftung erh�lt Einzug



%%%%%%%%%%%%%%%%%%%%%%%%%%%%%%%%%%%%%%%%%%%%%%%%%%%%%%%%%%%%%%%%%%%%%%%%%%%%%%%
% Seiteneinstellungen
%%%%%%%%%%%%%%%%%%%%%%%%%%%%%%%%%%%%%%%%%%%%%%%%%%%%%%%%%%%%%%%%%%%%%%%%%%%%%%%
% Dokumentstadium
\KOMAoptions{draft=false}  % Erleichert das erkennen von Fehlern im Entwurfsstadium

% Papierformat
\KOMAoptions{paper=A4}  % Format (Bei direkten Massangaben: 5cm:3cm)
%\KOMAoptions{paper=landscape}  % Ausrichtung  (Standard Portrait)
\KOMAoptions{pagesize=automedia}  % Angabe f�r Ausgangstreiber

% Bindekorrektur
%\KOMAoptions{BCOR=0.1cm}  % Bindekorrektur (Bereich der durchs Binden verloren geht)

% Gr�sse des Satzspiegels
\KOMAoptions{DIV=11}  % Gr�sse des Satzspiegels (Faktor ab 4), siehe S.38 scrguide.pdf; nur f�r A4 existieren Voreinstellungen, sonst calc o. classic als Option

% Randbereich
\KOMAoptions{mpinclude=false}  % Definiert ob der Randbereich zum Textk�rper hinzugez�hlt werden soll oder nicht -> TRUE nur bei Sonderf�llen

% Doppelspaltiges Dokument
\KOMAoptions{twocolumn=false}

% Absatzabstand
\KOMAoptions{parskip=true}

% Seitenstyle
\pagestyle{scrheadings}  % myheadings scrheadings

% Platzierzung letzte Zeile
\raggedbottom  % Letze Zeile liegt dort wo sie gerade ist -> unterschiedlicher vertikaler Abstand zu Blatt Ende (unerw�nscht bei doppelseitigem Druck)
%\flushbottom  % Letze Zeile immer am Schluss -> evtl. unterschiedliche Absatzabst�nde



%%%%%%%%%%%%%%%%%%%%%%%%%%%%%%%%%%%%%%%%%%%%%%%%%%%%%%%%%%%%%%%%%%%%%%%%%%%%%%%
% Kopf-/Fusszeile
%%%%%%%%%%%%%%%%%%%%%%%%%%%%%%%%%%%%%%%%%%%%%%%%%%%%%%%%%%%%%%%%%%%%%%%%%%%%%%%
% Kopfzeile
%\clearscrheadfoot  % Alle Vorgaben l�schen (plain und scrheadings)
\renewcommand*{\chaptermarkformat}{  % headmark ohne Kapitelnummer
%\chapappifchapterprefix{\ }\thechapter\autodot\enskip
}
% i: Links; c: Zentrum; o:Rechts; [Auf Kapitelstartseiten]; {Auf allen anderen Seiten}
\ihead[]{\textcolor{bfhblue}{\headmark}}  % Chapter Titel in Kopfzeile
\chead[]{}
\ohead[\textcolor{bfhblue}{\Titel, v\Version}]{\textcolor{bfhblue}{\Titel, v\Version}}  % Titel mit Version in Kopfzeile

% Fusszeile
\ifoot[\textcolor{bfhblue}{Berner Fachhochschule}]{\textcolor{bfhblue}{Berner Fachhochschule}}  % plain und scrheadings mit Dokumenttitel versehen
\cfoot[]{}
\ofoot[\textcolor{bfhblue}{\pagemark}]{\textcolor{bfhblue}{\pagemark}}  % plain und scrheadings mit Seitennummer versehen



%%%%%%%%%%%%%%%%%%%%%%%%%%%%%%%%%%%%%%%%%%%%%%%%%%%%%%%%%%%%%%%%%%%%%%%%%%%%%%%
% Fussnote
%%%%%%%%%%%%%%%%%%%%%%%%%%%%%%%%%%%%%%%%%%%%%%%%%%%%%%%%%%%%%%%%%%%%%%%%%%%%%%%
% Fussnote
%\KOMAoptions{footnotes=multiple}  % Fussnotennummern durch "," trennen



% Satzspiegel neu berechnen -> falls andere Schriftengeladen werden und/oder Zeilenabstand ver�ndert wird
\recalctypearea



%%%%%%%%%%%%%%%%%%%%%%%%%%%%%%%%%%%%%%%%%%%%%%%%%%%%%%%%%%%%%%%%%%%%%%%%%%%%%%%
% Paket Listings Konfiguration
%%%%%%%%%%%%%%%%%%%%%%%%%%%%%%%%%%%%%%%%%%%%%%%%%%%%%%%%%%%%%%%%%%%%%%%%%%%%%%%

% XML
\lstdefinestyle{XML}{numbers=left,
	basicstyle=\scriptsize\ttfamily,
	numberstyle=\tiny,
	xleftmargin=0.5\leftmargin,
	xrightmargin=\rightmargin,
	numbersep=5pt,
	backgroundcolor=\color{listingbackground},
	breaklines=true,
	captionpos=b,
	language=XML}

% MATlAB
\lstdefinestyle{Matlab}{numbers=left,
	basicstyle=\scriptsize\ttfamily,
	numberstyle=\tiny,
	xleftmargin=0.5\leftmargin,
	xrightmargin=\rightmargin,
	numbersep=5pt,
	backgroundcolor=\color{listingbackground},
	identifierstyle=\color{colIdentifier},
	keywordstyle=\color{colKeys},
	stringstyle=\color{colString},
	commentstyle=\color{colComments},
	breaklines=true,
	captionpos=b,
	language=Matlab}

% ANSI C
\lstdefinestyle{C}{numbers=left,
	basicstyle=\scriptsize\ttfamily,
	numberstyle=\tiny,
	xleftmargin=0.5\leftmargin,
	xrightmargin=\rightmargin,
	numbersep=5pt,
	backgroundcolor=\color{listingbackground},
	identifierstyle=\color{colIdentifier},
	keywordstyle=\color{colKeys},
	stringstyle=\color{colString},
	commentstyle=\color{colComments},
	breaklines=true,
	captionpos=b,
	language=[ANSI] C}


\makeindex  % Ab hier Indexieren
%%%%%%%%%%%%%%%%%%%%%%%%%%%%%%%%%%%%%%%%%%%%%%%%%%%%%%%%%%%%%%%%%%%%%%%%%%%%%%%
% Dokument Anfang
%%%%%%%%%%%%%%%%%%%%%%%%%%%%%%%%%%%%%%%%%%%%%%%%%%%%%%%%%%%%%%%%%%%%%%%%%%%%%%%
\begin{document}
%
%
%%%%%%%%%%%%%%%%%%%%%%%%%%%%%%%%%%%%%%%%%%%%%%%%%%%%%%%%%%%%%%%%%%%%%%%%%%%%%%%
% Titelseite
%%%%%%%%%%%%%%%%%%%%%%%%%%%%%%%%%%%%%%%%%%%%%%%%%%%%%%%%%%%%%%%%%%%%%%%%%%%%%%%
% Seitenzahl umschalten auf eigene Nummerierung: T-1, T-2, ...
\setcounter{page}{1}
\renewcommand{\thepage}{T-\arabic{page}}
% Titelseite einf�gen
\input{titlepage/titlepage}
\thispagestyle{empty}
% Leere Seite nach Titelseite einf�gen
\cleardoublepage
%
%
%
%\hypersetup{pageanchor=true}
\pagenumbering{roman}  % R�misch Nummerieren ab hier
%
%
%
%%%%%%%%%%%%%%%%%%%%%%%%%%%%%%%%%%%%%%%%%%%%%%%%%%%%%%%%%%%%%%%%%%%%%%%%%%%%%%%
% Abstract + Selbstaendige Arbeit
%%%%%%%%%%%%%%%%%%%%%%%%%%%%%%%%%%%%%%%%%%%%%%%%%%%%%%%%%%%%%%%%%%%%%%%%%%%%%%%
%%%%%%%%%%%%%%%%%%%%%%%%%%%%%%%%%%%%%%%%%%%%%%%%%%%%%%%%%%%%%%%%%%%%%%%%%%%%%%%
% Titel:   Abstract
% Autor:   Nicola K�ser
%%%%%%%%%%%%%%%%%%%%%%%%%%%%%%%%%%%%%%%%%%%%%%%%%%%%%%%%%%%%%%%%%%%%%%%%%%%%%%%

%%%%%%%%%%%%%%%%%%%%%%%%%%%%%%%%%%%%%%%%%%%%%%%%%%%%%%%%%%%%%%%%%%%%%%%%%%%%%%%
\chapter*{Abstract}

\chapter*{Selbständige Arbeit}
\label{ch:Selbständige Arbeit}
Ich erkläre ausdrückich, dass es sich bei dieser von mir eingereichten Arbeit um eine von mir selbst und ohne unerlaubte Beihilfe sowie in eigenen Worten verfasste Originalarbeit handelt.
Ich bestätige überdies, dass die Arbeit als Ganze oder in Teilen weder bereits einmal zur Abgeltung anderer Studienleistungen an der Berner Fachhochschule oder an einer anderen Universität oder Ausbildungseinrichtung eingereicht worden ist noch inskünftig durch mein Zutun als Abgeltung einer weiteren Studienleistung eingereicht werden wird.
Ich erkläre ausdrücklich, dass ich sämtliche in der oben genannten Arbeit enthaltenen Bezüge auf fremde Quellen als solche kenntlich gemacht haben.
\vspace{2cm}
\begin{tabbing}
xxxxxxxxxxxxxxxxxxxx\=xxxxxxxxxxxxxxxxxxxxxxx \kill
Ort, Datum      \>  \Ort, \Datum \\ \\ \\

Vorname Name    \> \AutorA \\  \\ 
Unterschrift    \> ......................................................... \\ \\  \\ 
\end{tabbing}

%
%
%
%%%%%%%%%%%%%%%%%%%%%%%%%%%%%%%%%%%%%%%%%%%%%%%%%%%%%%%%%%%%%%%%%%%%%%%%%%%%%%%
% �bersichts- & Inhaltsverzeichnis
%%%%%%%%%%%%%%%%%%%%%%%%%%%%%%%%%%%%%%%%%%%%%%%%%%%%%%%%%%%%%%%%%%%%%%%%%%%%%%%
% �bersichtsverzeichnis
\ifthenbool{createShorttoc}{
	% �bersicht mit definierter Tiefe
	\shorttoc{�bersicht}{0}
}
%
% Inhaltsverzeichnis
%\KOMAoptions{toc=listof}   % Abbildungs- und Tabellenverzeichnis ins Inhaltsverzeichnis
\KOMAoptions{toc=index}     % Stichwortverzeichnis ins Inhaltsverzeichnis
%
% Tiefe der Nummerierung
\setcounter{secnumdepth}{3}
% Tiefe der Auflistung im Inhaltsverzeichnis
\setcounter{tocdepth}{\tocmaxdepth}
%
% Inhaltsverzeichnis linksb�ndig
%\KOMAoptions{toc=flat}
%
% Verzeichnisse mit einer Kapitelnummer versehen
%\KOMAoptions{toc=listofnumbered}
%
% Inhaltsverzeichnis
\tableofcontents
%
\clearpage  % Seite beenden
%
%
%
%%%%%%%%%%%%%%%%%%%%%%%%%%%%%%%%%%%%%%%%%%%%%%%%%%%%%%%%%%%%%%%%%%%%%%%%%%%%%%%
% Dokumentinhalt
%%%%%%%%%%%%%%%%%%%%%%%%%%%%%%%%%%%%%%%%%%%%%%%%%%%%%%%%%%%%%%%%%%%%%%%%%%%%%%%
% Dokument arabisch Nummerieren
\pagenumbering{arabic}
%
% Kapitel 0: Beispiele
%\input{content/0_beispiele}
%
% Einleitung
%%%%%%%%%%%%%%%%%%%%%%%%%%%%%%%%%%%%%%%%%%%%%%%%%%%%%%%%%%%%%%%%%%%%%%%%%%%%%%%
% Titel:   Beispiele
% Autor:   Nicola K�ser
%%%%%%%%%%%%%%%%%%%%%%%%%%%%%%%%%%%%%%%%%%%%%%%%%%%%%%%%%%%%%%%%%%%%%%%%%%%%%%%
\chapter{Einleitung}\label{ch:einleitung}

% Einf�hrung ins Thema, allgemeinverst�ndlich, Referenzen, Erl�uterung der Aufgabe
%
% Hauptteil
%%%%%%%%%%%%%%%%%%%%%%%%%%%%%%%%%%%%%%%%%%%%%%%%%%%%%%%%%%%%%%%%%%%%%%%%%%%%%%%
% Titel:   Beispiele
% Autor:   Nicola K�ser
%%%%%%%%%%%%%%%%%%%%%%%%%%%%%%%%%%%%%%%%%%%%%%%%%%%%%%%%%%%%%%%%%%%%%%%%%%%%%%%
\chapter{Hauptteil}\label{ch:hauptteil}

% nachvollziehbare Entscheide und Entwicklungen, logischer Aufbau
%
% Ergebnisse
\input{content/3_ergebnis}
%
% Pendenzen
%%%%%%%%%%%%%%%%%%%%%%%%%%%%%%%%%%%%%%%%%%%%%%%%%%%%%%%%%%%%%%%%%%%%%%%%%%%%%%%
% Titel:   Beispiele
% Autor:   Nicola K�ser
%%%%%%%%%%%%%%%%%%%%%%%%%%%%%%%%%%%%%%%%%%%%%%%%%%%%%%%%%%%%%%%%%%%%%%%%%%%%%%%
\chapter{Pendenzen}\label{ch:pendenzen}

% M�ngel, Pendenzen, Stand der Arbeit, Ausblick und Empfehlungen
%
% Schlussfolgerung
%%%%%%%%%%%%%%%%%%%%%%%%%%%%%%%%%%%%%%%%%%%%%%%%%%%%%%%%%%%%%%%%%%%%%%%%%%%%%%%
% Titel:   Beispiele
% Autor:   Nicola K�ser
%%%%%%%%%%%%%%%%%%%%%%%%%%%%%%%%%%%%%%%%%%%%%%%%%%%%%%%%%%%%%%%%%%%%%%%%%%%%%%%
\chapter{Schlusswort}\label{ch:schlusswort}
%
%
%
%%%%%%%%%%%%%%%%%%%%%%%%%%%%%%%%%%%%%%%%%%%%%%%%%%%%%%%%%%%%%%%%%%%%%%%%%%%%%%%
% Hilfs�bersichten
%%%%%%%%%%%%%%%%%%%%%%%%%%%%%%%%%%%%%%%%%%%%%%%%%%%%%%%%%%%%%%%%%%%%%%%%%%%%%%%
\KOMAoptions{listof=leveldown}  % Verzeichnisse als Section statt Kapitel
%
\ifthenbool{createLists}{
	% Kapitel ohne Nummerierung, aber in Inhaltsverzeichnis
	\chapternn{Abbildungs- und Tabellen�bersicht}
	% Abbildungsverzeichnis
	\listoffigures
	%
	% Tabellenverzeichnis
	\listoftables
}
%
%
%
%%%%%%%%%%%%%%%%%%%%%%%%%%%%%%%%%%%%%%%%%%%%%%%%%%%%%%%%%%%%%%%%%%%%%%%%%%%%%%%
% Quellen
%%%%%%%%%%%%%%%%%%%%%%%%%%%%%%%%%%%%%%%%%%%%%%%%%%%%%%%%%%%%%%%%%%%%%%%%%%%%%%%
% �berpr�fen ob mindestens 1x \cite verwendet wurde, sonst Kapitel weglassen
\ifnum\value{citenum}>0
	% Kapitel ohne Nummerierung, aber in Inhaltsverzeichnis
	\chapternn{Quellen}
	In diesem Abschnitt sind alle Quellen des Dokumentes verzeichnet. Wird keine Referenz auf eine Quelle angegeben, so handelt es sich um selbst erarbeitete Inhalte des Autoren.
	\par citenum = \arabic{citenum}
	%
	% Literatur
	\printbibliography[heading=lit,keyword=lit,prefixnumbers=Lit,resetnumbers=true]
	%
	% Abbildungen
	\printbibliography[heading=pic,keyword=pic,prefixnumbers=Abb,resetnumbers=true]
	%
	% Online
	\printbibliography[heading=url,keyword=url,prefixnumbers=URL,resetnumbers=true]
\fi
%
%
%
%%%%%%%%%%%%%%%%%%%%%%%%%%%%%%%%%%%%%%%%%%%%%%%%%%%%%%%%%%%%%%%%%%%%%%%%%%%%%%%
% Stichwortverzeichnis
%%%%%%%%%%%%%%%%%%%%%%%%%%%%%%%%%%%%%%%%%%%%%%%%%%%%%%%%%%%%%%%%%%%%%%%%%%%%%%%
\ifthenbool{createIndex}{
	\renewcommand{\indexname}{Stichwortverzeichnis}
	\printindex
}
%
%
%
%%%%%%%%%%%%%%%%%%%%%%%%%%%%%%%%%%%%%%%%%%%%%%%%%%%%%%%%%%%%%%%%%%%%%%%%%%%%%%%
% Anhang
%%%%%%%%%%%%%%%%%%%%%%%%%%%%%%%%%%%%%%%%%%%%%%%%%%%%%%%%%%%%%%%%%%%%%%%%%%%%%%%
% Anhang im Inhaltsverzeichnis zu unterst auf der Seite anzeigen
\clearpage  % Damit Seitennummerierung richtig
\addtocontents{toc}{\vfill}
\appendix
% Keine Nummerierung (Buchstabe) anzeigen vor Kapitelname --> Manuell "Anhang A: bla"
\renewcommand{\thechapter}{}
% Seitenzahl umschalten auf alphabetisches Nummerieren
%\pagenumbering{alph}
% Seitenzahl umschalten auf eigene Nummerierung: A-1, A-2, ...
\renewcommand{\thepage}{A-\arabic{page}}
% Seitenz�hler auf 0, damit die Seite 1 des Anhangs dann auch A-1
\setcounter{page}{0}
% Anhang A einf�gen
%%%%%%%%%%%%%%%%%%%%%%%%%%%%%%%%%%%%%%%%%%%%%%%%%%%%%%%%%%%%%%%%%%%%%%%%%%%%%%%
% Titel:   Anhang
% Autor:   Nicola Käser
%%%%%%%%%%%%%%%%%%%%%%%%%%%%%%%%%%%%%%%%%%%%%%%%%%%%%%%%%%%%%%%%%%%%%%%%%%%%%%%

\chapter{Zeitplan}\label{ch:anhang_a}
    \todo{Zeitplan mit Tobias' Software}
    \includepdf[scale=1,page=-]{appendix/anhangA_Zeitplan.pdf}  % Zeitplan
% Seitenzahl umschalten auf eigene Nummerierung: B-1, B-2, ...
\renewcommand{\thepage}{B-\arabic{page}}
% Seitenz�hler auf 0, damit die Seite 1 des Anhangs dann auch B-1
\setcounter{page}{0}
% Anhang B einf�gen
%%%%%%%%%%%%%%%%%%%%%%%%%%%%%%%%%%%%%%%%%%%%%%%%%%%%%%%%%%%%%%%%%%%%%%%%%%%%%%%
% Titel:   Anhang
% Autor:   Nicola K�ser
%%%%%%%%%%%%%%%%%%%%%%%%%%%%%%%%%%%%%%%%%%%%%%%%%%%%%%%%%%%%%%%%%%%%%%%%%%%%%%%
\chapter{Anhang B: Bla}\label{ch:anhang_b}
	%\includepdf[scale=1,pages=-]{appendix/anhangB_Reglement.pdf}  % Reglement
%
%
%
\end{document}