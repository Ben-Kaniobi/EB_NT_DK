\documentclass{scrartcl}

\usepackage[ansinew]{inputenc}  % Schriftcodierung f�r Windows
\usepackage[T1]{fontenc}        % Schriftkodierung
\usepackage[ngerman]{babel}     % Neue de. Rechtschr. (Worttrennung, ���� etc.) als Parameter f�r babel
\usepackage{lmodern}            % Darstellung
\usepackage{wasysym}            % Spezielle Symbole
\usepackage{todonotes}          % ToDo-Notizen
\usepackage{perpage}
%\usepackage{calc}               % Erweiterte Berechnungen/Text messen
%\usepackage{amsmath}  % Math
%\usepackage[final]{pdfpages}   % PDF einbinden

\title{Testprotokoll Rangefinder}
\subtitle{Blackboxtest, Version 1.2}
\author{N. K�ser}
\date{1. Mai 2014} %\today

\MakePerPage{footnote}

\newcommand{\hardwareinfo}{STM32 F4 Discovery Board mit RoboBoard}
\newcommand{\hex}[1]{#1\textsubscript{hex}}%h}}%(16)}}

% Checkboxen
\newcommand{\platz}{\textcolor{white}{$\Box$}}
\newcommand{\leer}{$\Box$}
\newcommand{\ok}{$\CheckedBox$}
\newcommand{\nok}[1]{#1}%\mbox{\ooalign{\leer\cr\hidewidth \textcolor{red}{#1}\hidewidth\cr}}}


% Punktstyles global �ndern
\renewcommand{\labelitemii}{$\circ$}
\renewcommand{\labelitemiii}{$\cdot$}


\begin{document}
\maketitle%\todo{Datum der Testdurchf�hrung}
%
\hfill Test Nr.: \setlength{\fboxsep}{1pt}\fbox{1}%\fbox{2}
%\section{Software Set\_SRF08\_Address}
	%An dieser Software wurde nichts ver�ndert, weshalb sie nicht nochmal getestet werden muss.
		%%
\section{Software Rangefinder}
	\subsection{Allgemein}
		\begin{itemize}
			\item Software kann auf der Hardware\footnote{\hardwareinfo} debuged werden\dotfill \ok
		\end{itemize}
	\subsection{Infrarot Interrupt}
		\begin{itemize}
			\item Sensoren l�sen bei Hindernis Alarm aus:
			\begin{itemize}
				\item Flag f�r vorne links\dotfill A\ok
				\item Flag f�r vorne rechts\dotfill \ok
				\item Flag f�r hinten links\dotfill \ok
				\item Flag f�r hinten rechts\dotfill \ok
			\end{itemize}
			\item Sensoren setzen ohne Hindernis Alarm zur�ck:
			\begin{itemize}
				\item Flag f�r vorne links\dotfill A\ok
				\item Flag f�r vorne rechts\dotfill \ok
				\item Flag f�r hinten links\dotfill \ok
				\item Flag f�r hinten rechts\dotfill \ok
			\end{itemize}
			\item Software funktioniert auch falls nicht alle Sensoren angeschlossen\dotfill \ok
			\item Mit Definition \texttt{RANGEFINDER\_ONLY\_FW} funktionieren nur die vorderen Sensoren\dotfill \ok
		\end{itemize}
		%
	\subsection{Ultraschall Task}
		\begin{itemize}
			\item System/FreeRTOS:
			\begin{itemize}
				\item Task wird zu Beginn \textit{suspended}\dotfill \ok
			\end{itemize}
			\item Sensoren l�sen bei Hindernis (n�her als in Header definiert) Alarm aus:
			\begin{itemize}
				\item Flag f�r vorne\dotfill \ok
				\item Flag f�r hinten\dotfill \ok
				\item Flag f�r Separation\dotfill \ok
			\end{itemize}
			\item Sensoren setzen ohne Hindernis Alarm zur�ck:
			\begin{itemize}
				\item Sensor vorne\dotfill \ok
				\item Sensor hinten\dotfill \ok
				\item Flag f�r Separation\dotfill \ok
			\end{itemize}
			\item Task funktioniert auch falls nicht beide Sensoren angeschlossen\dotfill \ok
			\item Mit Definition \texttt{RANGEFINDER\_ONLY\_FW} funktioniert nur der vordere Sensor\dotfill \ok
		\end{itemize}
		%
	%\subsection{System/FreeRTOS}
	%
	%\clearpage
	\subsection{Notizen}
		\begin{enumerate}
			% Enumerationsstyle �ndern
			\renewcommand{\labelenumi}{\Alph{enumi}.}
			\item Beim Sensor vorne links wurde kein Alarm ausgel�st. Die Ursache war aber eine Besch�digung des Infrarot-Prints bei der Montage. Dieses Problem wurde gleich behoben und der Test konnte sogleich fortgesetzt werden.
		\end{enumerate}
		%
	\subsection{Fazit}
	Der Test wurde, bis auf ein kleines Hardware-Problem welches gleich behoben werden konnte, erfolgreich abgeschlossen.
%		\begin{enumerate}
%			% Enumerationsstyle �ndern
%			\renewcommand{\labelenumi}{\Alph{enumi}.}
%			\item blabla
%		\end{enumerate}
		%
\end{document}