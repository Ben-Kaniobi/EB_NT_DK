\documentclass{scrartcl}

\usepackage[ansinew]{inputenc}  % Schriftcodierung f�r Windows
\usepackage[T1]{fontenc}        % Schriftkodierung
\usepackage[ngerman]{babel}     % Neue de. Rechtschr. (Worttrennung, ���� etc.) als Parameter f�r babel
\usepackage{lmodern}            % Darstellung
\usepackage{wasysym}            % Spezielle Symbole
\usepackage{todonotes}          % ToDo-Notizen
\usepackage{perpage}
%\usepackage{calc}               % Erweiterte Berechnungen/Text messen
%\usepackage{amsmath}  % Math
%\usepackage[final]{pdfpages}   % PDF einbinden

\title{Testprotokoll Rangefinder}
\subtitle{Blackboxtest Version 1.0}
\author{N. K�ser}
\date{14. Januar 2014}

\MakePerPage{footnote}

\newcommand{\hardwareinfo}{STM32 F4 Discovery Board mit RoboBoard}
\newcommand{\hex}[1]{#1\textsubscript{hex}}%h}}%(16)}}

% Checkboxen
\newcommand{\platz}{\textcolor{white}{$\Box$}}
\newcommand{\leer}{$\Box$}
\newcommand{\ok}{$\CheckedBox$}
\newcommand{\nok}[1]{#1}%\mbox{\ooalign{\leer\cr\hidewidth \textcolor{red}{#1}\hidewidth\cr}}}


% Punktstyles global �ndern
\renewcommand{\labelitemii}{$\circ$}
\renewcommand{\labelitemiii}{$\cdot$}


\begin{document}
\maketitle%\todo{Datum der Testdurchf�hrung}
%
\hfill Test Nr.: \setlength{\fboxsep}{1pt}\fbox{1}\fbox{2}
\section{Software Set\_SRF08\_Address}
	\subsection{Allgemein}
		\begin{itemize}
			\item Software kann auf der Hardware\footnote{\hardwareinfo} debuged werden\dotfill \ok\platz%
		\end{itemize}
	\subsection{SRF08 Adresse �ndern}
		Nach erfolgreicher Adress�nderung leuchtet die LED des Ultraschallmoduls ununterbrochen. Die aktuelle Adresse ist dann durch den Blinkcode beim erneuten Einstecken erkennbar.
		\begin{itemize}
			\item Falsche Adresse wird nicht geschrieben
			\begin{itemize}
				\item Nicht im g�ltigen Bereich:
				\begin{itemize}
					\item \hex{00}\dotfill \ok\platz
					\item \hex{DE}\dotfill \ok\platz
				\end{itemize}
				\item Ungerade:
				\begin{itemize}
					\item \hex{E5}\dotfill \ok\platz
					\item \hex{FB}\dotfill \ok\platz
				\end{itemize}
			\end{itemize}
			\item G�ltige Adresse wird geschrieben:
			\begin{itemize}
				\item \hex{FE}\dotfill \ok\platz
				\item \hex{E0}\dotfill \ok\platz
			\end{itemize}
			\item Erneute Adress�nderung ist m�glich ($\Rightarrow$ neue Adresse funktioniert)\dotfill \ok\platz
		\end{itemize}
		%
		\clearpage
		%
\section{Software Rangefinder}
	\subsection{Allgemein}
		\begin{itemize}
			\item Software kann auf der Hardware\footnote{\hardwareinfo} debuged werden\dotfill \ok\ok
		\end{itemize}
	\subsection{Infrarot Task}
		\begin{itemize}
			\item Sensoren l�sen bei Hindernis Alarm aus:
			\begin{itemize}
				\item Sensor links\dotfill \nok{A}\ok
				\item Sensor mitte\dotfill \nok{A}\ok
				\item Sensor rechts\dotfill \nok{A}\ok
				\item Sensor hinten\dotfill \ok\ok
			\end{itemize}
			\item Sensoren setzen ohne Hindernis Alarm zur�ck:
			\begin{itemize}
				\item Sensor links\dotfill \ok\ok
				\item Sensor mitte\dotfill \ok\ok
				\item Sensor rechts\dotfill \ok\ok
				\item Sensor hinten\dotfill \ok\ok
			\end{itemize}
			\item Task funktioniert auch falls nicht alle Sensoren angeschlossen\dotfill \ok\ok
		\end{itemize}
		%
	\subsection{Ultraschall Task}
		\begin{itemize}
			\item Sensoren l�sen bei Hindernis (n�her als definiert) Alarm aus:
			\begin{itemize}
				\item Sensor vorne\dotfill \ok\ok
				\item Sensor hinten\dotfill \ok\ok
			\end{itemize}
			\item Sensoren setzen ohne Hindernis Alarm zur�ck:
			\begin{itemize}
				\item Sensor vorne\dotfill \ok\ok
				\item Sensor hinten\dotfill \ok\ok
			\end{itemize}
			\item Task funktioniert auch falls nicht alle Sensoren angeschlossen\dotfill \nok{B}\nok{B}
		\end{itemize}
		%
	\clearpage
	\subsection{Notizen}
		\begin{enumerate}
			% Enumerationsstyle �ndern
			\renewcommand{\labelenumi}{\Alph{enumi}.}
			\item Bei einem Hindernis der Sensoren wurde kein Alarm ausgel�st.
			\item Angeschlossener Sensor liefert immer gleichen Wert.
		\end{enumerate}
		%
	\subsection{Fazit}
		\begin{enumerate}
			% Enumerationsstyle �ndern
			\renewcommand{\labelenumi}{\Alph{enumi}.}
			\item Es wurden OR- statt AND-Verkn�pfungen gebraucht. (Die IR-Sensoren sind low-aktiv\footnote{HIGH: Kein Objekt detektiert; LOW: Objekt detektiert}.) Fehler wurde behoben.
			\item In der I\textsuperscript{2}C-Implementation der Vorg�nger\footnote{File "`src/lib/i2c.c"'} werden mit \textit{while}-Schlaufen die Buszust�nde gepr�fft. Dies konnte zu Endlosschlaufen f�hren. Zum Beheben wurden schon vor dem Test Timeout-Variablen hinzugef�gt. Auch manuelles L�schen des Busy-Flags\footnote{Das Busy-Flag ist gesetzt w�hrend der Bus gerade f�r eine �bertragung verwendet wird.} w�re in diesem Fall notwendig, das Flag kann aber nur gelesen werden (read only). Der Fehler kann somit nicht behoben werden. Dies ist jedoch nicht allzu schlimm, da eigentlich sowieso immer Beide Sensoren angeschlossen sein sollten.
		\end{enumerate}
		%
\end{document}






