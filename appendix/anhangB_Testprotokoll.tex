\documentclass{scrartcl}

\usepackage[ansinew]{inputenc}  % Schriftcodierung f�r Windows
\usepackage[T1]{fontenc}        % Schriftkodierung
\usepackage[ngerman]{babel}     % Neue de. Rechtschr. (Worttrennung, ���� etc.) als Parameter f�r babel
\usepackage{lmodern}            % Darstellung
\usepackage{wasysym}            % Spezielle Symbole
\usepackage{todonotes}          % ToDo-Notizen
%\usepackage{calc}               % Erweiterte Berechnungen/Text messen
%\usepackage{amsmath}  % Math
%\usepackage[final]{pdfpages}   % PDF einbinden

\title{Testprotokoll Rangefinder}
\subtitle{Blackboxtest Version 1.0}
\author{N. K�ser}
\date{z.B. 6. Januar 2014}%\today}

\newcommand{\hardwareinfo}{STM32 F4 Discovery Board mit RoboBoard}

\begin{document}
\maketitle%\todo{Datum der Testdurchf�hrung}
%
\hfill Test Nr.: \setlength{\fboxsep}{1pt}\fbox{1}\fbox{2}
\section{Software Set\_SRF08\_Address}
	\subsection{Allgemein}
		\begin{itemize}
			\item Software kann auf der Hardware\footnote{\hardwareinfo} debuged werden\dotfill $\Box$$\Box$%$\CheckedBox$
		\end{itemize}
	\subsection{SRF08 Adresse �ndern}
		Die aktuelle Adresse eines Moduls ist durch den Blinkcode beim Einstecken erkennbar. So kann jeweils �berpr�ft werden, ob eine Adress�nderung erfolgreich war.
		\begin{itemize}
		% Punktstyles �ndern
		\renewcommand{\labelitemii}{$\circ$}
		\renewcommand{\labelitemiii}{$\cdot$}
			\item Falsche Adresse wird nicht geschrieben
			\begin{itemize}
				\item Nicht im g�ltigen Bereich:
				\begin{itemize}
					\item 0x00\dotfill $\Box$$\Box$%$\CheckedBox$
					\item 0xDE\dotfill $\Box$$\Box$%$\CheckedBox$
				\end{itemize}
				\item Ungerade:
				\begin{itemize}
					\item 0xE5\dotfill $\Box$$\Box$%$\CheckedBox$
					\item 0xFB\dotfill $\Box$$\Box$%$\CheckedBox$
				\end{itemize}
			\end{itemize}
			\item G�ltige Adresse wird geschrieben:
			\begin{itemize}
				\item 0xFE\dotfill $\Box$$\Box$%$\CheckedBox$
				\item 0xE0\dotfill $\Box$$\Box$%$\CheckedBox$
			\end{itemize}
			\item Neue Adresse kann danach verwendet werden:
			\begin{itemize}
				\item 0xFE\dotfill $\Box$$\Box$%$\CheckedBox$
				\item 0xE0\dotfill $\Box$$\Box$%$\CheckedBox$
			\end{itemize}
		\end{itemize}
		%
		\clearpage
		%
\section{Software Rangefinder}
	\subsection{Allgemein}
		\begin{itemize}
			\item Software kann auf der Hardware\footnote{\hardwareinfo} debuged werden\dotfill $\Box$$\Box$%$\CheckedBox$
		\end{itemize}
	\subsection{Infrarot Task}
		\begin{itemize}
			\item Sensoren l�sen bei Hindernis Alarm aus:
			\begin{itemize}
				\item Sensor links\dotfill $\Box$$\Box$%$\CheckedBox$
				\item Sensor mitte\dotfill $\Box$$\Box$%$\CheckedBox$
				\item Sensor rechts\dotfill $\Box$$\Box$%$\CheckedBox$
				\item Sensor hinten\dotfill $\Box$$\Box$%$\CheckedBox$
			\end{itemize}
			\item Sensoren setzen ohne Hindernis Alarm zur�ck:
			\begin{itemize}
				\item Sensor links\dotfill $\Box$$\Box$%$\CheckedBox$
				\item Sensor mitte\dotfill $\Box$$\Box$%$\CheckedBox$
				\item Sensor rechts\dotfill $\Box$$\Box$%$\CheckedBox$
				\item Sensor hinten\dotfill $\Box$$\Box$%$\CheckedBox$
			\end{itemize}
			\item Task funktioniert auch falls nicht alle Sensoren angeschlossen\dotfill $\Box$$\Box$%$\CheckedBox$
		\end{itemize}
		%
	\subsection{Ultraschall Task}
		\begin{itemize}
			\item Sensoren l�sen bei Hindernis (n�her als definiert) Alarm aus:
			\begin{itemize}
				\item Sensor vorne\dotfill $\Box$$\Box$%$\CheckedBox$
				\item Sensor hinten\dotfill $\Box$$\Box$%$\CheckedBox$
			\end{itemize}
			\item Sensoren setzen ohne Hindernis Alarm zur�ck:
			\begin{itemize}
				\item Sensor vorne\dotfill $\Box$$\Box$%$\CheckedBox$
				\item Sensor hinten\dotfill $\Box$$\Box$%$\CheckedBox$
			\end{itemize}
			\item Task funktioniert auch falls nicht alle Sensoren angeschlossen\dotfill $\Box$$\Box$%$\CheckedBox$
		\end{itemize}
		%
	\subsection{Notizen}
		A. Bla bla
		%
	\subsection{Fazit}\todo{Evtl. mit Notizen zusammen wegen Platz}
		Das Verhalten bei Notiz A war voraussehbar und st�rt die Bedienbarkeit nicht. Wird ignoriert.
		%
\end{document}






